\chapter{Clock}

Muitos eventos, quando tratamos de música, são dependentes do tempo.
O Pure Data, segundo o autor, é perfeitamente sincronizado no tempo podendo,
obviamente, ter problemas com o tempo real já que o mundo real não é perfeitamente
sincronizado.
Para sincronizar eventos discretos, o PD utiliza, internamente, seu relógio DSP.
Isto significa que, todos os eventos são agendados para ocorrer em um determinado
bloco de amostras podendo o usuário ajustar a precisão destes eventos alterando
o tamanho do bloco de processamento ou a taxa de amostragem.
Para evitar que cada \external utilize um relógio próprio para agendar eventos
discretos, o PD conta com funções específicas para tratar o tempo.


