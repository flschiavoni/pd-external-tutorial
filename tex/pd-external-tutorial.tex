\documentclass[10pt,a4paper]{report}

% localização com hifenização
\usepackage[utf8]{inputenc}
\usepackage[brazil]{babel}
\usepackage{graphicx}
\usepackage{epsfig}
\usepackage{cite}
\usepackage{textcomp} %para o greater than
\usepackage[margin=1in]{geometry}

\usepackage{enumerate}
\usepackage{xspace}
\usepackage{colortbl}
\usepackage{listings}             % Include the listings-package
\usepackage{todonotes}

\usepackage{comment}

\usepackage{url}      % Para URL com link
\usepackage{hyperref} % Para sumario com link

\newcommand{\Mysize}{0.6}

\lstset{numbers=left, stepnumber=1, frame=single, rulesepcolor=\color{blue}, language=C}

\graphicspath{{./../images/}}

% commands
\newcommand{\externals}{\emph{externals}\xspace}
\newcommand{\Externals}{\emph{Externals}\xspace}
\newcommand{\external}{\emph{external}\xspace}
\newcommand{\External}{\emph{External}\xspace}
\newcommand{\patch}{\emph{patch}\xspace}
\newcommand{\patches}{\emph{patches}\xspace}
\newcommand{\code}[1]{\hrulefill\texttt{\\#1\\\indent}}

% listings configuration
\lstset{ %
language=C,                % the language of the code
basicstyle=\ttfamily,       % the size of the fonts that are used for the code
numbers=left,                   % where to put the line-numbers
numberstyle=\footnotesize,      % the size of the fonts that are used for the line-numbers
%stepnumber=2,                   % the step between two line-numbers. If it's 1, each line 
                                % will be numbered
numbersep=8pt,                  % how far the line-numbers are from the code
backgroundcolor=\color{white},  % choose the background color. You must add \usepackage{color}
showspaces=false,               % show spaces adding particular underscores
showstringspaces=false,         % underline spaces within strings
showtabs=false,                 % show tabs within strings adding particular underscores
frame=single,                   % adds a frame around the code
tabsize=2,                      % sets default tabsize to 2 spaces
captionpos=b,                   % sets the caption-position to bottom
breaklines=true,                % sets automatic line breaking
breakatwhitespace=false,        % sets if automatic breaks should only happen at whitespace
title=\lstname,                 % show the filename of files included with \lstinputlisting;
                                % also try caption instead of title
escapeinside={\%*}{*)},         % if you want to add a comment within your code
xleftmargin={2em},
xrightmargin={2em},
columns=fixed,
morekeywords={*,...}            % if you want to add more keywords to the set
}

\begin{document}

% ---------------------------------------------------------------------------- %
% CAPA
% Nota: O título para as dissertações/teses do IME-USP devem caber em um
% orifício de 10,7cm de largura x 6,0cm de altura que há na capa fornecida pela SPG.
\thispagestyle{empty}
\begin{center}
    \vspace*{2.3cm}
    \vskip 8cm
    \textbf{\Large{Tutorial prático para o desenvolvimento de \externals em C para o
Pure Data}}\\

    \vspace*{1.2cm}
    \vskip 5cm
    \Large{Flávio Luiz Schiavoni - André Bianchi}

    \vskip 5cm

    \vskip 0.5cm
    \normalsize{São Paulo, \today}
\end{center}

\tableofcontents

% chapters
\part{Primeiros passos}
% ----------------------------------------------------------------------------
% Introdução
% ----------------------------------------------------------------------------
 
\chapter{Introdução}

Pure Data, ou simplesmente Pd, é um ambiente visual de programação musical que permite a criação de
aplicações musicais complexas a partir da combinação de componentes visuais
mais simples chamados \textbf{objetos}. As distribuições oficiais do Pure Data
contêm diversos objetos prontos para o uso, mas também permitem a extensão de
suas funcionalidades através da criação de novos objetos utilizando C/C++.
Desta forma, novas linhas de código escritas pelo usuário são compilados como
bibliotecas dinâmicas e podem ser carregadas pelo programa em tempo de
execução. Objetos desta forma levam o nome de \textbf{\externals}.

Este é um tutorial prático para o desenvolvimento de \externals em C para o
Pure Data. A iniciativa de escrever este documento surgiu no primeiro semestre
de 2011, durante a disciplina de Computação Musical ministrada pelo professor
Marcelo Gomes de Queiroz no Instituto de Matemática e Estatística da
Universidade de São Paulo. A intenção deste tutorial é auxiliar programadores
a desenvolver \externals de maneira bastante simples através de exemplos
práticos.

Mais do que ampliar a gama de objetos do Pure Data e criar novos objetos, o
objetivo deste trabalho é também fornecer ao pesquisador de computação musical
uma ferramenta para implementar e testar algoritmos de processamento de áudio
para caráter de estudo. Isto significa que podemos reimplementar várias coisas
que já existem no Pure Data simplesmente porque é didático programar e colocar
algoritmos para funcionar.

\section{Escrevendo \externals}

O código fonte do Pure Data é organizado de acordo com convenções de
programação orientada a objetos. Para o desenvolvimento de \externals, é
necessário seguir estas convenções e fornecer ao ambiente uma nova classe com
alguns métodos específicos, como veremos mais adiante. Para desenvolver para o
Pure Data, é necessário importar o arquivo de cabeçalho
\texttt{m\_pd.h}\footnote{http://pure-data.git.sourceforge.net/git/gitweb.cgi?p=pure-data/pure-data;a=blob\_plain;f=src/m\_pd.h;hb=HEAD},
que contém definições de constantes, tipos e funções.

Uma boa fonte de informação é o tutorial de
\externals\footnote{http://iem.at/pd/externals-HOWTO/pd-externals-HOWTO.pdf}
escrito pelo IOHannes\footnote{http://puredata.info/author/zmoelnig}, um dos
programadores do Pure Data. Apesar de ter utilizado este documento como ponto
de partida, boa parte do que está incluso no presente tutorial foi aprendido a
partir da leitura do código-fonte de \externals contidos no repositório
oficial do Pure
Data\footnote{http://pure-data.svn.sourceforge.net/viewvc/pure-data/trunk/externals/}.

Navegando pelos códigos-fonte deste repositório você poderá notar que os programadores
que escreveram os externals que hoje estão disponível para o Pd seguiram estas convenções
e por isto a leitura destes códigos-fonte pode ser didática e simples.

Por esta razão, o primeiro conselho que damos para quem irá escrever \externals é seguir
estas convenções, mesmo que as mesmas não sejam a maneira como você está acostumado a 
programar deste jeito pois assim seu código também será didático e simples de entender.

Este tutorial não pretende cobrir os algoritmos de processamento de sinais mas explicar
como implementar estes algoritmos como objetos do Pd. Para processamento de sinais há
uma vasta bibliografia disponível que possui os algoritmos e o ferramental matemático
necessário para sua implementação.

\todo{Será que podemos citar aqui algum livro ou material para DSP?}
 

\section{Organização do código-fonte e do objeto compilado}
\label{sec:organizacao}

Um novo \external corresponde a uma nova classe na arquitetura orientada a
objetos do Pure Data. Para que o carregamento da biblioteca dinâmica
em tempo de execução funcione corretamente, é necessário que o
arquivo binário produzido possua o mesmo nome que a classe correspondente ao
\external.

Para criar, por exemplo, um \external chamado ``passa-baixas", podemos
escrever seu código-fonte em um arquivo chamado \texttt{passa-baixas.c}, e em
seguida compilar um objeto de biblioteca compartilhada chamado
\texttt{passa-baixas.pd\_linux}, no caso do sistema GNU/Linux. Outras
arquiteturas de sistema utilizam outras extensões para o nome do objeto com a
biblioteca compartilhada do \external, como por exemplo \texttt{.dll} (M\$
Windows), \texttt{.pd\_irix5} (SGI Irix) ou \texttt{.pd\_darwin} (Mac OS X).

\textbf{Importante:} O nome do arquivo com o código-fonte não possui formato
obrigatório, mas o nome do objeto compilado com a biblioteca dinâmica deve
sempre corresponder ao nome da classe, assim como sua extensão deve sempre
corresponder à arquitetura do sistema utilizado.

O mesmo cuidado é recomendado para os métodos que serão definidos internamente
no objeto. Os nomes de métodos que serão apresentados neste material seguem o padrão
encontrado no repositório do Pd. É fortemente recomendado que o mesmo padrão seja
utilizado em seu texto.

\section{Compilação}
\label{sec:compiling}

Para criar um objeto binário que pode ser carregado no Pure Data em tempo de
execução, primeiro compilamos o código fonte, criando assim um ou mais objetos
intermediários, e em seguida utilizamos um ligador (\emph{linker}) para criar
um objeto de biblioteca compartilhada.

No GNU/Linux, uma forma de realizar o processo
\texttt{example01.c} $\rightarrow$ \texttt{example01.o} $\rightarrow$
\texttt{example01.pd\_linux} é a seguinte:

\vspace{1em}
\begin{lstlisting}
EXTNAME=example01
cc -DPD -fPIC -Wall -o ${EXTNAME}.o -c ${EXTNAME}.c
ld -shared -lc -lm -o ${EXTNAME}.pd_linux ${EXTNAME}.o
rm ${EXTNAME}.o
\end{lstlisting}

A opção de compilação \texttt{-fPIC} resulta na criação de código binário que
roda independente de sua posição na memória, adequado para geração de
bibliotecas compartilhadas. A opção \texttt{-shared} passada para o ligador
determina a criação de uma biblioteca compartilhada.

Para facilitar a compilação, é interessante utilizar um \texttt{makefile}. Os
exemplos deste tutorial estão acompanhadas de um \texttt{makefile} produzido
pelo professor Marcelo Queiroz e adaptado para este tutorial.

Tem uma dezena de jeito de compilar pro Windows, usando o Mingw ou o C++ Builder.
Aqui\footnote{http://puredata.hurleur.com/sujet-1029-problem-compiling-external-windows}
 temos exemplos e muitas discussões de como compilar externals no Windows.

\todo{Será que compensa testarmos isto tudo?}

\subsection{Misturando código C e C++}

Existem algumas diferenças entre compiladores C e C++ que tornam a sintaxe das
linguagens incompatíveis, gerando resultados diferentes para um mesmo trecho
de código. Um exemplo disso que influencia o funcionamento de \externals no
Pure Data é a geração da tabela de símbolos dos objetos binários.

Compiladores C++ realizam um processo chamado \emph{name mangling} (ou
``dilaceramento de nomes"), que consiste em alterar o nome de funções,
estruturas, classes, etc, incluindo informações sobre o espaço de nomes do
objeto em questão. Isto resulta em nome diferentes gravados nas tabelas de
símbolos dos objetos binários, o que pode confundir o Pure Data no momento do
carregamento de um \external.

Para garantir que um compilador C++ gere nomes compatíveis com objetos
binários C, utilize a expressão \texttt{extern "C"} na frente dos nomes das
funções que serão chamadas pelo Pure Data:

\begin{lstlisting}
extern "C" example01_setup(void);
extern "C" example01_new(void);
\end{lstlisting}


\section{Arquivos de ajuda}

É importante distribuir, junto com novos \externals, um arquivo de ajuda do
Pure Data com instruções e exemplos de utilização. Como convenção, o arquivo
de ajuda deve ter o mesmo nome que o \external, acrescido do sufixo
\texttt{-help.pd}. Por exemplo, para o código fonte \texttt{example01.c}, que
gera o objeto \texttt{example01.pd\_linux}, escrevemos também o arquivo
\texttt{example01-help.pd}.

No próximo capítulo veremos uma forma de associar o arquivo de ajuda com a
opção de ajuda que aparece no menu contextual com um clique do botão direito
no objeto do external dentro do Pure Data.

\section{Utilizando \externals}
\label{sec:using}

Para carregar um \external em um \texttt{patch} do Pure Data em tempo de
execução, basta criar um objeto (com \texttt{CTRL+1} ou acessando o menu
\texttt{Put} $\rightarrow$ \texttt{Object}) com o caminho (relativo ou
absoluto) para o objeto compilado com a biblioteca compartilhada, omitindo a
extensão.

É possível adicionar o diretório que contém o arquivo binário do \external ao
caminho de busca do Pure Data, de forma que para acessá-lo de dentro de um
\patch não seja necessário digitar o caminho inteiro até o objeto. Isto pode
ser feito através da passagem de um parâmetro na linha de comando do Pure Data
com a opção \texttt{-path <caminho>}, ou de forma gráfica acessando a opção
\texttt{File} $\rightarrow$ \texttt{Path...} no menu do Pure Data,
como pode ser visto na figura \ref{fig:search-path}.

\begin{figure}[h!]
  \centering
  \includegraphics[width=0.7\textwidth]{path}
  \caption{Adicionando o diretório de um \external ao caminho de busca do Pure Data.}
  \label{fig:search-path}
\end{figure}

Para carregar uma biblioteca de \externals (mais de um \external no mesmo
arquivo-fonte), é possível indicar o nome da bibliotecai na linha
de comando do Pure Data utilizando a opção \texttt{-lib <biblioteca>}, ou
também graficamente através do menu \texttt{File} $\rightarrow$
\texttt{Startup...}, como pode ser visto na figura \ref{fig:lib}.

\begin{figure}[h!]
  \centering
  \includegraphics[width=0.7\textwidth]{startup}
  \caption{Adicionando uma biblioteca ao Pure Data.}
  \label{fig:lib}
\end{figure}


% ----------------------------------------------------------------------------
% O básico de um \external
% ----------------------------------------------------------------------------

\chapter{O básico de um \external}

Escrever um \external significa seguir as recomendações da API. Peço ao leitor
bastante paciência pois este tutorial pretende andar um pouco devagar para
mostrar os passos da escrita de um \external.

\section{Um \external simples}

Como dissemos anteriormente, a arquitetura do Pure Data é organizada de acordo
com o paradigma de orientação a objetos: cada objeto gráfico do Pure Data
corresponde a uma instância de uma classe. Neste sentido, um \external está
associado a um conjunto de estruturas de dados que representam classes em C.
Para cada classe é necessário haver métodos de instanciação, destruição,
processamento de sinais, tratamento de mensagens, etc.

A infraestrutura mínima para o funcionamento de um external (de nome, digamos,
\texttt{<external>}) consiste em uma estrutura de dados para a representação
de uma classe, que deve ter nome \texttt{t\_<external>}, e dois métodos
obrigatórios, chamados \texttt{<external>\_setup()} e
\texttt{<external>\_new()}. Note que a convenção de nomes utilizada no Pure
Data é de que toda função deve ser nomeada da forma
\texttt{<contexto>\_<funcao>()}.

A estutura de dados que representa uma classe do Pure Data deve
obrigatoriamente possuir o primeiro atributo do tipo \texttt{t\_object}, no
qual é armazenado o objeto criado no momento da instanciação.  Outros
atributos podem ser adicionados a esta estrutura de maneira que cada instância
da mesma classe possua os atributos necessários para seu funcionamento. Uma
classe que acessa um arquivo, por exemplo, pode possuir como atributos uma
string para guardar o caminho e um inteiro para guardar o descritor do
arquivo.

Um exemplo de estrutura de dados para representação de uma classe chamada
\texttt{example1} consiste no seguinte:

\vspace{1em}
\begin{lstlisting}
static t_class *example1_class;

typedef struct _example1 {
  t_object x_obj;
} t_example1;
\end{lstlisting}

Sempre que um \external é carregado pelo Pure Data, o método de nome
\texttt{<external>\_setup()} é executado. No exemplo dado acima, o Pure
Data irá procurar, no arquivo binário \texttt{example1.pd\_linux} que contém
a biblioteca compartilhada, o método de nome \texttt{example1\_setup(void)}.
Este método é utilizado para realizar a inicialização da classe, informando ao
Pure Data da existência de uma nova classe no sistema e associando a ela os
métodos de instanciação e destruição, além de outras informações:

\vspace{1em}
\begin{lstlisting}
void example1_setup(void) {
  example1_class = class_new(
    gensym("example1"),         // Nome simbolico
    (t_newmethod) example1_new, // Construtor
    0,                          // Destrutor
    sizeof (t_example1),        // Tamanho dos atributos
    CLASS_NOINLET,              // Flags da classe
    0                           // Tipos dos argumentos
  );
}
\end{lstlisting}

Dentro do método \texttt{<external>\_setup()} não há limite para o número de
classes a definir, de forma que é possível definir apenas uma classe (como no
exemplo 1) ou uma biblioteca inteira com várias classes (como no exemplo 3).
A introdução de uma nova classe no sistema é realizada pela função
\texttt{class\_new()}. São parâmetros da função \texttt{class\_new()}:

\begin{itemize}
\item Nome simbólico da classe.
\item Método construtor de um objeto.
\item Método destrutor de um objeto.
\item Tamanho do espaço de dados dos atributos de um objeto.
\item Flags que definem a representação gráfica do objeto.
\item Tipos dos parâmetros a serem passados para o construtor quando da
      instanciação de um objeto (veja o próximo capítulo).
\end{itemize}

É necessário terminar a lista de tipos de parâmetros com um número inteiro 0,
para indicar ao Pure Data que a lista de tipos terminou. Consulte a
documentação da função \texttt{class\_new()} para mais
detalhes\footnote{http://pdstatic.iem.at/externals-HOWTO/node9.html\#SECTION00092100000000000000}.

O método \texttt{<external>\_new()}, que foi associado como método de
instanciação de objetos na chamada de \texttt{class\_new()}, realiza a
instanciação de objetos propriamente dita. Neste método, além da instanciação
de um novo objeto através da função \texttt{pd\_new()}, é possível definir os
valores dos atributos da estrutura de dados da classe e também inicializar
quaisquer outros contextos que sejam necessários, como por exemplo abrir
arquivos, preencher vetores, alocar memória, etc.

\vspace{1em}
\begin{lstlisting}
// Construtor da classe
void * example1_new(void) {
    t_example1 *x = (t_example1 *) pd_new(example1_class);
    return (void *) x;
}
\end{lstlisting}

Após a criação da estrutura de dados dos métodos da forma mencionada acima, a
compilação realizada da forma descrita na seção \ref{sec:compiling}, e a
criação do objeto no Pure Data como descrito na seção \ref{sec:using}, o
resultado pode ser visto na figura \ref{fig:example01working}.

\begin{figure}[h!]
  \centering
  \includegraphics[width=0.7\textwidth]{example1}
  \caption{Nosso primeiro \external do PD. Ainda inútil. :-$\left.\right)$}
  \label{fig:example01working}
\end{figure}

\section{Uma biblioteca simples}

Um mesmo método \texttt{<external>\_setup()} pode definir várias classes
diferentes. A isto damos o nome de biblioteca. Neste cenário, o método
\texttt{<external>\_setup()} possui o mesmo nome do arquivo com a biblioteca,
mas cada classe podem ter um nome diferente (veja o exemplo 3).

\vspace{1em}
\begin{lstlisting}
void example3_setup(void) {
  post("Initializing my library");

  myobj1_class = class_new(
    gensym("myobj1"),
    (t_newmethod) myobj1_new, // Constructor
    0,
    sizeof (t_myobj1),
    CLASS_NOINLET,
    0);
  class_sethelpsymbol(myobj1_class, gensym("myobj1-help"));

  myobj2_class = class_new(
    gensym("myobj2"),
    (t_newmethod) myobj2_new, // Constructor
    0,
    sizeof (t_myobj2),
    CLASS_NOINLET,
    0);
  class_sethelpsymbol(myobj2_class, gensym("myobj2-help"));
}
\end{lstlisting}

Se o arquivo foi preenchido corretamente, compilado corretamente e adicionado
ao caminho do PureData, teremos o resultado visto na figura \ref{fig:exemplo3}.

\begin{figure}[h!]
	\centering
	\includegraphics[width=0.7\textwidth]{example3}
	\caption{Nosso segundo \external do PD. Ainda inútil. :-$\left.\right)$}
        \label{fig:exemplo3}
\end{figure}


Dentro do Pure Data, um clique com o botão direito em um objeto abre um menu
no qual uma das opções é \texttt{Ajuda}. Quando esta opção é selecionada, o
Pure Data abre um patch associado ao objeto, que deve conter instruções e
exemplos de uso. Por padrão, o Pure Data procura um arquivo com o mesmo nome
que o external (acrescido da extensão \texttt{-help.pd}) no diretório padrão de
documentação (\texttt{doc/5.reference}). Para associar um arquivo diferente do
padrão, basta utilizar a função \texttt{class\_sethelpsymbol}:

\vspace{1em}
\begin{lstlisting}
class_sethelpsymbol(myclass_class, gensym("my_class-help"));
\end{lstlisting}

Um objeto pode ainda ter outros nomes (\emph{aliases}). Para definir isto
podemos utilizar a função \texttt{class\_addcreator()}. Veja o exemplo:

\vspace{1em}
\begin{lstlisting}
class_addcreator((t_newmethod)medusa_new, gensym("med"), 0);
\end{lstlisting}

\section{Variáveis globais}

É possível utilizar variáveis globais para armazenar dados de um \external.
Estas variáveis são visíveis para todas as intâncias de objetos do \external e
todas podem alterar seus valores. Isto pode ser útil ou um desastre (veja o
exemplo16). Por exemplo, cada instância do \external \texttt{example16}
definido a partir do código a seguir incrementa em uma unidade o valor do
contador, como pode ser visto na figura \ref{fig:counter}:

\vspace{1em}
\begin{lstlisting}
int count = 0;

void * example16_new(void) {
    t_example16 *x = (t_example16 *) pd_new(example16_class);
    post("Counter value: %d",count);
    count++;
    return (void *) x;
}
\end{lstlisting}

\begin{figure}[h!]
  \centering
  \includegraphics[width=0.7\textwidth]{example16}
  \caption{Repare na saída da janela principal.}
  \label{fig:counter}
\end{figure}

Caso isto não seja desejável, o ideal é incluir as variáveis dentro da
estrutura do objeto. Assim, neste exemplo cada instância terá seu próprio
contador:

\vspace{1em}
\nopagebreak{
\begin{lstlisting}
static t_class *example_class;

typedef struct _example {
    t_object x_obj;
    t_int counter;
} t_example;

void * example_new(void) {
    t_example *x = (t_example *) pd_new(example_class);
    post("Counter value: %d",x->counter);
    x->counter++;
    return (void *) x;
}
\end{lstlisting}
}

% ----------------------------------------------------------------------------
% OS TIPOS DE DADOS DO PD
% ----------------------------------------------------------------------------

\chapter{Os tipos de dados do PD}

Uma vez que o Pure Data é utilizado em diversas plataformas, muitos tipos
comuns de variáveis, como \texttt{int}, são redefinidos. Para escrever um
\external que seja portável para qualquer plataforma, é razoável que você
utilize os tipos providos pelo Pure Data. Como dissemos na seção
\ref{sec:organizacao}, para escrever um \external, é necessário incluir o
arquivo \texttt{m\_pd.h} que possui definições de constantes (versão do Pure
Data, sistema operacional, compilador, etc), estruturas, assinaturas de
funções e tipos de dados.

Existem muitos tipos predefinidos que devem fazer a vida do programador mais
simples. Em geral, os tipos do pd têm nome iniciado por \texttt{t\_}.

\begin{center}
\begin{tabular}{|l|l|}
\hline
pd-type & description \\
\hline
\texttt{t\_atom} & \texttt{atom} \\
\texttt{t\_float} & \texttt{floatingpoint value} \\
\texttt{t\_symbol} & \texttt{symbol} \\
\texttt{t\_gpointer} & \texttt{pointer (to graphical objects)} \\
\texttt{t\_int} & \texttt{integer value} \\
\texttt{t\_signal} & \texttt{structure of a signal} \\
\texttt{t\_sample} & \texttt{audiosignal-value (floatingpoint)} \\
\texttt{t\_outlet} & \texttt{outletof an object} \\
\texttt{t\_inlet} & \texttt{inlet of an object} \\
\texttt{t\_object} & \texttt{object-interna} \\
\texttt{t\_class} & \texttt{a pd-class} \\
\texttt{t\_method} & \texttt{class-method} \\
\texttt{t\_newmethod} & \texttt{pointer to a constructor (new-routine)} \\
\hline
\end{tabular}
\end{center}

\section{Símbolos}

Um símbolo corresponde a um valor constante de uma \emph{string}, ou seja, a
uma sequência de letras que formam uma palavra.

Cada símbolo é armazenado em uma tabela de busca por razões de performance. A
função \texttt{gensym(char *)} procura por uma string em uma tabela de busca e
retorna o endereço daquele símbolo. Se a string não foi encontrada na tabela,
um novo símbolo é adicionado.


\texttt{t\_symbol} e \texttt{*gensym}

\section{Mensagens}

Dados que não correspondem a áudio são distribuídos via um sistema de
mensagens. Cada mensagem consiste em um ``seletor" e uma lista de átomos.

\subsection{Átomos}

Existem três tipos de átomos:

\begin{itemize}
\item \texttt{A\_FLOAT}: um valor numérico (de ponto flutuante).
\item \texttt{A\_SYMBOL}: um valor simbólico (string).
\item \texttt{A\_POINTER}: um ponteiro.
\end{itemize}

Valores numéricos são sempre considerados valores de ponto flutuante
(\texttt{t\_float}), mesmo que possam ser exibidos como valores inteiros.

Átomos do tipo \texttt{A\_POINTER} não são muito importantes (para \externals
simples).
Since pd is used on several plattforms, many ordinary types of variables, like int, are re-defined. To write portable code, it is reasonable to use types provided by pd.

O tipo de um átomo \texttt{a} é armazenado no elemento da estrutura
\texttt{a.a\_type}.

\subsection{Seletores}

Um seletor é um símbolo que define o tipo de uma mensagem. Existe cinco
seletores pré-definidos:

\begin{itemize}
  \item \texttt{bang}: rotula um gatilho de evento. Uma mensagem de
    \texttt{bang} consiste somente do seletor e não contém uma lista de átomos.
  \item \texttt{float} rotula um valor numérico. A lista de uma mensagem
    \texttt{float} contém um único átomo de tipo \texttt{A\_FLOAT}.
  \item \texttt{symbol} rotula um valor simbólico. A lista de uma mensagem
    \texttt{symbol} consiste em um único átimo do tipo \texttt{A\_SYMBOL}.
  \item \texttt{pointer} rotula um valor de ponteiro. A lista de uma mensagem do
    tipo \texttt{pointer} contém um único átimo do tipo \texttt{A\_POINTER}.
  \item \texttt{list} rotula uma lista de um ou mais átomos de tipos arbitrários.
\end{itemize}

Uma vez que os símbolos para estes seletores são utilizados com frequência,
seu endereço na tabela de símbolos pode ser utilizado diretamente, sem a
necessidade da utilização de \texttt{gensym}:

\begin{center}
\begin{tabular}{|l|l|l|}
\hline
  selector & lookup-routine & lookup-address \\
\hline
  \texttt{bang} & \texttt{gensym("bang")} & \texttt{\&s\_bang} \\
  \texttt{float} & \texttt{gensym("float")} & \texttt{\&s\_float} \\
  \texttt{symbol} & \texttt{gensym("symbol")} & \texttt{\&s\_symbol} \\
  \texttt{pointer} & \texttt{gensym("pointer")} & \texttt{\&s\_pointer} \\
  \texttt{list} & \texttt{gensym("list")} & \texttt{\&s\_list} \\
  \texttt{--}  (signal) & \texttt{gensym("signal")} & \texttt{\&s\_symbol} \\
\hline
\end{tabular}
\end{center}

Outros seletores também podem ser utilizados. A classe receptora tem que
prover um médodo para um seletor específico ou para \texttt{anything}, que
corresponde a qualquer seletor arbitrário.

Mensagens que não possuem seletor explícito e começam com um valor numérico
são reconhecidas automaticamente como mensagens \texttt{float} (se consistirem
de apenas um átomo) ou como mensagens \texttt{list} (se forem compostas de
diversos átomos).

Por exemplo, as mensagens \texttt{12.429} e \texttt{float 12.429} são
idênticas. Da mesma forma, as mensagens \texttt{list 1 para voce} é idêntica a
\texttt{1 para voce}.


% ----------------------------------------------------------------------------
% CONSTRUTOR E DESTRUTOR
% ----------------------------------------------------------------------------

\chapter{Construtor e destrutor}

O Construtor de um objeto pode receber parâmetros. Estes parâmetros são
ilustrados abaixo.

\begin{figure}[h!]
	\centering
	\includegraphics[width=0.7\textwidth]{example2}
	\caption{External recebendo parâmetros. Note a tela de saída no fundo da imagem.}
\end{figure}

\section{Construtor}

Parâmetros de inicialização no construtor podem permitir que inicializemos o
external com determinados valores. Isto é feito definindo os parâmetros no
métodos class\_new() quanto na definição da função construtora. (Veja o
exemplo02).

\begin{lstlisting}

// Constructos of the class
void * example2_new(t_symbol * arg1, t_floatarg arg2) {
    t_example2 *x = (t_example2 *) pd_new(example2_class);
    post("First arg: %s", arg1->s_name);
    post("Second arg: %f", arg2);
    return (void *) x;
}

void example2_setup(void) {
    example2_class = class_new(gensym("example2"),
            (t_newmethod) example2_new, // Constructor
            0,
            sizeof (t_example2),
	    CLASS_NOINLET,
            A_DEFFLOAT, // First Constructor parameter
            A_DEFSYMBOL, // Second Consctructo parameter
            0);
}
\end{lstlisting}

Notem que os parâmetros são definidos com um tipo e são recebidos com outro.
São tipos padrões do Pure Data. Estes tipos são chamados atom e alguns estão
documentados no tutorial do IOHannes. Eles são:

\begin{itemize}
\item A\_NULL,
\item A\_FLOAT,
\item A\_SYMBOL,
\item A\_POINTER,
\item A\_SEMI,
\item A\_COMMA,
\item A\_DEFFLOAT,
\item A\_DEFSYM,
\item A\_DOLLAR, 
\item A\_DOLLSYM,
\item A\_GIMME,
\item A\_CANT
\end{itemize}
(Retirado do arquivo m\_pd.h)

*** Nunca usei todos eles. Pra que será que servem?

Entre estes tipos, um deles pode aceitar qualquer quantidade e tipo de
parâmetros. É o A\_GIMME. (Veja o exemplo09). 

\begin{lstlisting}

// Constructos of the class
void * example9_new(t_symbol *s, int argc, t_atom * argv) {
    t_example9 *x = (t_example9 *) pd_new(example9_class);
    post("%d parameters received",argc);
    return (void *) x;
}


void example9_setup(void) {
    example9_class = class_new(gensym("example9"),
            (t_newmethod) example9_new, // Constructor
            (t_method) example9_destroy, // Destructor
            sizeof (t_example9),
	    CLASS_NOINLET,
	    A_GIMME, // Allows various parameters
            0); // LAST argument is ALWAYS zero
}
\end{lstlisting}

Quando utilizamos o A\_GIMME a função construtora trabalha como um programa
main em C. Recebe os parâmetros argv e argc, um com uma lista de atoms e outro
com a quantidade de atoms nesta lista.

\begin{figure}[h!]
	\centering
	\includegraphics[width=0.7\textwidth]{example9}
	\caption{Diferente da linguagem C, o primeiro parâmetro não é o nome do external.}
\end{figure}

Note que o Pure Data não obriga que o usuário passe parâmetros para o objeto. É
como se todo construtor, independentemente de como ele está definido, aceitasse
sua instanciação vazia. Cabe ao programador verificar se os parâmetros
recebidos são em quantidade, tipo e valor esperado e, caso não seja, abortar a
construção do objeto e não retornar sua instância.

\section{Destrutor}
O destrutor de uma classe permite liberar a memória do Pure Data dos dados que foram alocados. (Veja o exemplo07)
\begin{lstlisting}
// Destroy the class
void example9_destroy(t_example9 *x) {
   post("You say good bye and I say hello");
}

void example9_setup(void) {
    example9_class = class_new(gensym("example9"),
            (t_newmethod) example9_new, // Constructor
            (t_method) example9_destroy, // Destructor
            sizeof (t_example9),
	    CLASS_NOINLET,
	    A_GIMME, // Allows various parameters
            0); // LAST argument is ALWAYS zero
}

\end{lstlisting}

A liberação da memória pode ser feita com a função freebytes() definida na API
do Pure Data.

\begin{lstlisting}
void freebytes(void *x, size_t nbytes)
\end{lstlisting}


\section{Inlets e Outlets}


\begin{frame}{Inlets e outlets}
\begin{figure}[ht!]
\centering
\includegraphics[height=0.6\textheight]{example1}
\caption{Para que serve isso!?}
\end{figure}
\end{frame}


\begin{frame}{Inlets passivos (1/3)}
\begin{figure}[h!]
\centering
\includegraphics[height=0.8\textheight]{example4}
\label{fig:inlet-passivo}
\end{figure}
\end{frame}


\begin{frame}[fragile]{Inlets passivos (2/3)}
\begin{lstlisting}
static t_class *example4_class;

typedef struct _example4 {
  t_object x_obj;
  t_float my_float;
} t_example4;

// Constructor of the class
void *example4_new(t_symbol *arg1, t_floatarg arg2) {
  t_example4 *x = (t_example4*)pd_new(example4_class);
  post("First arg: %s", arg1->s_name);
  post("Second arg: %f", arg2);
  floatinlet_new(&x->x_obj, &x->my_float);
  return (void *) x;
}
\end{lstlisting}
\end{frame}


\begin{frame}{Inlets passivos (3/3)}
As funções para criar inlets passivos dos tipos mais comuns são:
\begin{itemize}
\item \texttt{floatinlet\_new(t\_object *owner, t\_float *fp)}
\item \texttt{symbolinlet\_new(t\_object *owner, t\_symbol **sp)}
\item \texttt{pointerinlet\_new(t\_object *owner, t\_gpointer *gp)}
\end{itemize}
\end{frame}


\begin{frame}{Inlets ativos (1/2)}
\begin{figure}[h!]
\centering
\includegraphics[width=0.7\textwidth]{example5}
\caption{Inlets ativos.}
\label{fig:inlet-ativo}
\end{figure}
\end{frame}


\begin{frame}[fragile]{Inlets ativos (2/2)}
\begin{lstlisting}[language=C]
// inlet-methods receive the object as first argument.
void example5_bang(t_example5 *x) { 
  post("BANGED!");
  post("My_float value: %f",x->my_float);
}

void example5_anything(t_example5 *x, t_symbol *s, int argc, t_atom *argv){
  post("ANYTHING!");
}

void example5_setup(void) {
  example5_class = class_new(gensym("example5"),
    (t_newmethod) example5_new, // Constructor
    0,  sizeof (t_example5), CLASS_DEFAULT,
    0); // LAST argument is ALWAYS zero
  class_addbang(example5_class, example5_bang);
  class_addanything(example5_class, example5_anything);
}
\end{lstlisting}
\end{frame}

\begin{frame}{Tratamento de mensagens (1/4)}
\begin{figure}[h!]
\centering
\includegraphics[width=0.7\textwidth]{example8}
\label{fig:inlet-ativo}
\end{figure}
\end{frame}

\begin{frame}[fragile]{Tratamento de mensagens (2/4)}
\lstinputlisting[name=Exemplo 08,linerange=47-61]{../examples/example08.c}
\end{frame}


\begin{frame}[fragile]{Tratamento de mensagens (3/4)}
\lstinputlisting[name=Exemplo 08,linerange=17-25,firstnumber=last]{../examples/example08.c}
\end{frame}


\begin{frame}[fragile]{Tratamento de mensagens (4/4)}
\lstinputlisting[name=Exemplo 08,linerange=27-45,firstnumber=last]{../examples/example08.c}
\end{frame}


\begin{frame}{Outlets (1/4)}
\begin{figure}[h!]
\centering
\includegraphics[width=0.7\textwidth]{example6}
\caption{Um external bem útil que recebe um bang e envia um bang.}
\label{fig:outlet-bang}
\end{figure}
\end{frame}


\begin{frame}{Outlets (2/4)}
\lstinputlisting[name=Exemplo 06,linerange=35-45]{../examples/example06.c}
\end{frame}


\begin{frame}{Outlets (3/4)}
\lstinputlisting[name=Exemplo 06,linerange=23-33,firstnumber=last]{../examples/example06.c}
\end{frame}


\begin{frame}{Outlets (4/4)}
\lstinputlisting[name=Exemplo 06,linerange=6-21,firstnumber=last]{../examples/example06.c}
\end{frame}


% ----------------------------------------------------------------------------
% DSP
% ----------------------------------------------------------------------------

\chapter{DSP}
 (Precisa mudar o nome dos exemplos para exemplo~. Neste caso, a função setup deve ser renomeada para "tilde\_setup")

Enfim chegamos no processamento de áudio propriamente dito. Digital Signal Processing ou processamento de sinal digital. O PD tem inlets especiais para o processamento de sinal. É fácil reconhecer. Eles são pintados de preto.
\begin{figure}[h!]
	\centering
	\includegraphics[width=0.7\textwidth]{example10}
	\caption{Primeiro Inlet DSP}
\end{figure}

\section{Primeiro inlet para DSP}
Para trabalharmos com DSP no Pure Data é necessário alguns cuidados. (Veja o
exemplo10) Primeiramente, temos de ter na estrutura de dados um atributo do
tipo t\_float para armazenar o valor de entrada do inlet.

\begin{lstlisting}
typedef struct _example10 {
    t_object x_obj;
    t_float x_f;
/* place to hold inlet's value if it's set by message */
} t_example10;
\end{lstlisting}

Caso trabalhemos com apenas um inlet de DSP, o mesmo pode utilizar o atributo
"mágico" do primeiro inlet da esquerda. Isto pode ser feito com a atribuição do
atributo ao inlet DSP no método setup(). Para isto temos de utilizar o tipo de
classe padrão do Pure Data (CLASS\_DEFAULT). Também é necessário definirmos
qual será o método chamado quando o DSP é iniciado. Este método é adicionado
como os métodos dos inlets vistos anteriormente tendo porém sua mensagem
associada ao tipo "dsp".

\begin{lstlisting}
void example10_setup(void) {
    example10_class = class_new(gensym("example10"),
            (t_newmethod) example10_new, // Constructor
            (t_method) example10_destroy, // Destructor
            sizeof (t_example10),
	    CLASS_DEFAULT,
	    A_GIMME, // Allows various parameters
            0); // LAST argument is ALWAYS zero

  /* this is magic to declare that the leftmost, "main" inlet
     takes signals; other signal inlets are done differently...*/
     CLASS_MAINSIGNALIN(example10_class, t_example10, x_f);

   // This method will add a signal inlet and associate a method to do this
    class_addmethod(example10_class, (t_method) example10_dsp, 
	gensym("dsp"), 0); 

}
\end{lstlisting}
 
A declaração de outros inlets DSP será vista logo adiante.

O próximo passo é definirmos o método DSP definido no setup().

\begin{lstlisting}
static void example10_dsp(t_example10 *x, t_signal **sp){
  dsp_add(example10_perform, 3, sp[0]->s_vec, sp[0]->s_n, x); 
}
\end{lstlisting}

O método associado ao DSP será chamado TODA VEZ QUE O DSP FOR INICIADO. Por
isto, cuidado com alocações de memória, inicialização de variáveis e estas
coisas. Neste método definiremos quem será chamado em cada laço de execução de
processamento de áudio do PD. Neste cado é a função example10\_perform. Este
método recebe o array de sinal que a conexão deste inlet traz. Este sinal está
na variável **sp. Na atribuição acima, passamos para o método perform os
atributos:

\begin{itemize}
\item método perform
\item quantidade de atributos do método
\item vetor de saída
\item tamanho do vetor (tamanho do bloco)
\item instância do nosso external
\end{itemize}

Podemos passar para o método perform quaisquer parâmetros em qualquer ordem. Só
é importante e óbvio que devemos lembrar quais parâmetros foram passados e em
qual ordem. O próximo passo é criar o método perform propriamente dito.

\begin{lstlisting}
static t_int * example10_perform(t_int *w){
   t_float *in = (t_float *)(w[1]);
   int n = (int)(w[2]);
   t_example10 *x = (t_example10 *)(w[3]);

  //(... DO SOMETHING)

  return (w + 4); // proximo bloco
}
\end{lstlisting}

O método perform receberá como parâmetro um array com os dados que definimos no
método anterior. Neste caso na posição 0 algo que eu não sei o que é, na
posição 1 o vetor de saída, na posição 2 o tamanho do vetor de saída e na
posição 3 a nossa estrutura de dados. Este método deve retornar a próxima
posição do vetor, ou seja, o atributo de entrada + quantidade de atributos do
método + 1.

\section{Vários Inlets DSP}
Podemos ter vários inlets de DSP no nosso external (Veja o exemplo 11). A
criação de inlets adicionais não é feita no método setup() mas sim no
construtor. Só será necessário criar o primeiro inlet se a classe não for do
tipo CLASS\_DEFAULT.

\begin{lstlisting}
// Constructos of the class
void * example11_new(t_symbol *s, int argc, t_atom * argv) {
    t_example11 *x = (t_example11 *) pd_new(example11_class);
    inlet_new(&x->x_obj, &x->x_obj.ob_pd, &s_signal, &s_signal); // second
    inlet_new(&x->x_obj, &x->x_obj.ob_pd, &s_signal, &s_signal); // third
    inlet_new(&x->x_obj, &x->x_obj.ob_pd, &s_signal, &s_signal); // fourth
    return (void *) x;
}
\end{lstlisting}

Nosso método class\_addmethod é exatamente igual ao anterior mas temos uma
mudança na quantidade de parâmetros por causa da quantidade de inlets.

\begin{lstlisting}
static void example11_dsp(t_example11 *x, t_signal **sp){
  dsp_add(example11_perform, 6, sp[0]->s_vec, sp[0]->s_n, x);
}
\end{lstlisting}

Note que precisamos agora alterar a quantidade de parâmetros passadas ao método
perform. O método perform ficará assim:
\begin{lstlisting}
static t_int * example11_perform(t_int *w){
   t_float *in1 = (t_float *)(w[1]);
   t_float *in2 = (t_float *)(w[2]);
   t_float *in3 = (t_float *)(w[3]);
   t_float *in4 = (t_float *)(w[4]);
   int n = (int)(w[5]);
   t_example11 *x = (t_example11 *)(w[6]);
   // DO SOMETHING...
  return (w + 7); // proximo bloco
}
\end{lstlisting}

Nosso external pronto deverá ter a seguinte aparência:
\begin{figure}[h!]
	\centering
	\includegraphics[width=0.7\textwidth]{example11}
	\caption{Vários inlets DSP.}
\end{figure}

\section{Primeiro outlet DSP}

A criação dos outlets é feita no construtor do external (Veja o exemplo12) e
não é necessário termos adicionado os outlets a estrutura da classe.

\begin{lstlisting}
// Constructos of the class
void * example12_new(t_symbol *s, int argc, t_atom * argv) {
    t_example12 *x = (t_example12 *) pd_new(example12_class);
    outlet_new(&x->x_obj, &s_signal); // first signal outlet
    outlet_new(&x->x_obj, &s_signal); // second signal outlet
    outlet_new(&x->x_obj, &s_signal); // third signal outlet
    outlet_new(&x->x_obj, &s_signal); // fourth signal outlet
    return (void *) x;
}
\end{lstlisting}

Sabendo que temos 4 outlets, a definição do nosso método perform será idêntica
ao da criação de 4 inlets.

\begin{lstlisting}
static void example12_dsp(t_example12 *x, t_signal **sp){
  dsp_add(example12_perform, 6, sp[0]->s_vec, sp[0]->s_n, x);
}
\end{lstlisting}

O método perform também será idêntico ao do exemplo com 4 inlets, porém o mesmo
receberá 4 outlets.

\begin{lstlisting}
static t_int * example12_perform(t_int *w){
   t_float *out1 = (t_float *)(w[1]);
   t_float *out2 = (t_float *)(w[2]);
   t_float *out3 = (t_float *)(w[3]);
   t_float *out4 = (t_float *)(w[4]);
   int n = (int)(w[5]);
   t_example12 *x = (t_example12 *)(w[6]);
  // DO SOMETHING
  return (w + 7); // proximo bloco
}
\end{lstlisting}

\begin{figure}[h!]
	\centering
	\includegraphics[width=0.7\textwidth]{example12}
	\caption{Primeiro Outlet DSP.}
\end{figure}

\section{Inlets e outlets DSP}
Nosso próximo exemplo (Veja o exemplo 13) mistura no mesmo objeto inlets e
outlets DSP. Uma coisa bastante comum. Acredito que seja óbvio a construção do
mesmo. Não precisamos associar estes inlets e outlets a nossa estrutura de
dados. Precisamos apenas criar os inlets e outlets no construtor (lembre-se que
o primeiro inlet já foi criado no método setup. Ele é mágico!).

\begin{lstlisting}
static t_int * example13_perform(t_int *w){
   t_float *in1 = (t_float *)(w[1]);
   t_float *in2 = (t_float *)(w[2]);
   t_float *in3 = (t_float *)(w[3]);
   t_float *in4 = (t_float *)(w[4]);
   t_float *out1 = (t_float *)(w[5]);
   t_float *out2 = (t_float *)(w[6]);
   t_float *out3 = (t_float *)(w[7]);
   t_float *out4 = (t_float *)(w[8]);
   int n = (int)(w[9]);
   t_example13 *x = (t_example13 *)(w[10]);
  return (w + 11); // proximo bloco
}
\end{lstlisting}

No método seguinte avisamos o método perform do tamanho dos dados.
\begin{lstlisting}
static void example13_dsp(t_example13 *x, t_signal **sp){
  dsp_add(example13_perform, 10, sp[0]->s_vec, sp[0]->s_n, x);
}
\end{lstlisting}

No método perform teremos primeiro os buffers de entrada e depois os buffers de
saída.

\begin{lstlisting}
static t_int * example13_perform(t_int *w){
   t_float *in1 = (t_float *)(w[1]);
   t_float *in2 = (t_float *)(w[2]);
   t_float *in3 = (t_float *)(w[3]);
   t_float *in4 = (t_float *)(w[4]);
   t_float *out1 = (t_float *)(w[5]);
   t_float *out2 = (t_float *)(w[6]);
   t_float *out3 = (t_float *)(w[7]);
   t_float *out4 = (t_float *)(w[8]);
   int n = (int)(w[9]);
   t_example13 *x = (t_example13 *)(w[10]);
   //DO IT!
  return (w + 11); // proximo bloco
}
\end{lstlisting}

\begin{figure}[h!]
	\centering
	\includegraphics[width=0.7\textwidth]{example13}
	\caption{Vários inlets e outlets DSP.}
\end{figure}

\section{Inlets e outlets DSP criados dinamicamente}

Podemos definir um parâmetro no construtor que nos diga a quantidade de inlets
e/ou de outlets DSP que um external terá. Neste caso, temos algumas
possibilidades. A primeira é usarmos uma variável para dizer quantos inlets e
outlets teremos na função dsp.

Exemplo 17

A segunda é usar outro método. Exemplo 18 - abordagem da medusa.



\part{Avançando...}
% ----------------------------------------------------------------------------
% MULTITHREADING
% ----------------------------------------------------------------------------

\chapter{Multithreading}

Como visto no capítulo anterior, o bloco de processamento do Pd possui um tempo máximo para
a execução. Caso tenhamos um processo tipo produtor / consumidor, por exemplo, que consuma
um tempo maior que o bloco possa permitir é ideal usarmos threads para isto.

A programação multithread não é exatamente comum no Pure Data mas pode ser útil para várias
coisas como escrita de arquivo, envio para a rede ou atualização na GUI (que veremos a seguir).

Apesar de existir várias bibliotecas que permitam a criação de programação paralela, como o 
comando fork do Linux, é desejável que os externals do Pure Data sejam compatíveis com outros
sistemas operacionais. Pesquisando entre os repositórios do autor, Miller Puckette, encontramos
que o mesmo utiliza threads POSIX da biblioteca pthread \footnote{Para maiores informações, visite:
 https://computing.llnl.gov/tutorials/pthreads/}.

Note que esta solução, que em muito se aproxima da última forma de criar inlets e outlets DSP
implica em não trabalharmos mais em tempo real. Implementações deste tipo não podem ser pensadas
para processamentos aonde a entrada de áudio será processada e devolvida na saída de áudio no
mesmo bloco de processamento do Pd.

\section{Criando threads}

Para utilizarmos a biblioteca de threads do POSIX é necessário importá-la. Depois disto, para 
termos controle sobre as threads que criamos, é necessário criarmos uma variável
que armazene esta thread (Veja o exemplo 20).

\begin{lstlisting}
#include <pthread.h>

...
typedef struct _example20 {
    t_object x_obj;
    pthread_t example20_thread;
} t_example20;

\end{lstlisting}

O próximo passo é criarmos uma função que seja associada a esta thread e a criação da thread 
propriamente dita. A criação da thread pode ser feita na função DSP. Isto implica criar a thread
e iniciá-la toda vez que o DSP do Pd for iniciado.

\begin{lstlisting}
void * example20_thread_function(void * arg) {
    t_example20 *x = (t_example20 *) arg;
    while(1){
            //DO SOMETHING
	    printf("Threading running!\n");
	    sleep(1);
	}
    return 0;
}

static void example20_dsp(t_example20 *x, t_signal **sp){
  pthread_create(&x->example20_thread, NULL, example20_thread_function, x);
  dsp_add(example20_perform, 1 , x);
}
\end{lstlisting}

A função de criação da thread irá receber a variável aonde a thread será armazenada (ID), 
o os atributos desta thread (no caso de NULL, serão o default)
\footnote{Para uma lista completa dos atributos, visite: 
http://sourceware.org/pthreads-win32/manual/pthread\_attr\_init.html} ,
 a função de inicialização associada a esta thread e os argumentos passados para esta função. 

Caso seja passado mais de um argumento, é recomendado que se crie uma estrutura de dados (struct)
e que a mesma seja passada como argumento para a thread.

\section{Gerenciamento de threads}

Há várias funções para o gerenciamento de uma thread definida na biblioteca pthread.h. Entre elas:
\begin{itemize}
\item pthread\_detach(threadid) - Indica para a implementação que o armazenamento da thread pode ser recuperado 
quando a mesma se encontra terminada
\item pthread\_join(threadid,status) - Indica para o trecho de código que chamou a thread que o mesmo deve esperar
que a mesma tenha terminado sua execução.
\item pthread\_exit(void *value\_ptr) - Encerra a execução de uma thread e libera sua alocação de memória.
\end{itemize}

Em princípio, threads POSIX não possuem funções para pausar e continuar a execução. Apesar disto, é possível 
implementar estes comandos por meio dos mutex que veremos a seguir.

\section{Controle de concorrência}

Uma das dificuldades de implementarmos com threads é controlar a concorrência por recursos entre threads.
Situação clara de Race Condition, é necessário que controlemos o acesso de threads concorrentes a trechos 
de código que acessam dados comuns. Isto é feito por meio de Mutex (mutual exclusion), sistemas de controle
atômicos que garantem que apenas uma thread será executada sobre um trecho de código por vez.

Imaginemos os Mutex definidos assim:

\begin{lstlisting}
pthread_mutex_t lock = PTHREAD_MUTEX_INITIALIZER;
pthread_cond_t cond = PTHREAD_COND_INITIALIZER;
int play = 0;
\end{lstlisting}

O controle ao trecho de código pode ser feito da seguinte forma:

\begin{lstlisting}
for(;;) { /* Playback loop */
    pthread_mutex_lock(&lock);
    while(!play) { /* We're paused */
        pthread_cond_wait(&cond, &lock); /* Wait for play signal */
    }
    pthread_mutex_unlock(&lock);
    /* Continue playback */
}
\end{lstlisting}

\todo{Creio que seria mais útil um exemplo funcional...}

%% ----------------------------------------------------------------------------
% ORIENTAÇÃO A OBJETOS
% ----------------------------------------------------------------------------

\chapter{Orientação a Objetos}

Parte do texto foi retirada de http://www.katjaas.nl/pitchshift/soundtouch~.html.

Este cara é um exemplo de external que utiliza uma biblioteca OO:
http://pure-data.svn.sourceforge.net/viewvc/pure-data/trunk/externals/fftw/fftw~.c?revision=6017&view=markup
No caso, a bibliteca utilizada é a FFTW3. Outro exemplo é o PDCUDA do Drebs!

Outro exemplo que utilizad OO em external é o pixopencv.

http://pure-data.svn.sourceforge.net/viewvc/pure-data/trunk/externals/pix_opencv/

É possível criar externals utilizando C++ e orientação a objetos. A parte crítica de tal implementação é garantir que o PD consiga enxergar as funções dentro do objeto compilado do C++. É necessário pensar nisto não apenas como uma alternativa para criar externals em C++ mas também para utilizar bibliotecas C++ para a criação de externals. Utilizar estas bibliotecas torna necessário que a mesma seja compilada na forma como símbolos em C. Para garantir isto é necessário utilizar a definição extern "C".

TERMINAR ESTE EXEMPLO. COMO FAZER PARA O MAKEFILE FUNCIONAR TAMBÉM PARA ARQUIVOS .H e .CPP?


\begin{lstlisting}
extern "C" void example18_setup(void) {
    example18_class = class_new(gensym("example18"),
            (t_newmethod) example18_new, // Constructor
            0,
            sizeof (t_example18),
	    CLASS_NOINLET,
            0);
};
\end{lstlisting}

Usando o setup "externalizado" o mesmo passa a ser exportado e compilado em um objeto com esta função "visível". Deve ser o suficiente para compilar o external com sua API utilizando C++. Para compilar, usados um compilador C++ como o g++ do Linux.

\begin{lstlisting}
LINUXCFLAGS = -msse -DPD -DUNIX -DICECAST -O3 -funroll-loops -fomit-frame-pointer -fcheck-new \
    -Wall -W -Wshadow \
    -Wno-unused -Wno-parentheses -Wno-switch -fvisibility=hidden

LINUXINCLUDE =  -I ./include

	g++ $(LINUXCFLAGS) $(LINUXINCLUDE) -c *.cpp 
	g++ --export-dynamic -shared -o $*.pd_linux *.o -lc -lm -lstdc++ 
	strip --strip-unneeded $(NAME).pd_linux
	rm -f *.o ../$(NAME).pd_linux
\end{lstlisting}


 % Não vale a pena gastar tempo com isto.
\chapter{Send e Receive}

A comunicação do PD com \externals nem sempre é feita por conexões explícitas.
Outra maneira de permitir a comunicação entre objetos é por meio de send e
receive.
Esta opção está presente em objetos gráficos como o toggle e bang, por exemplo.
Para utilizar tal tipo de mensagem é necessário definir o nome da mensagem que
o objeto pretende receber.

\begin{figure}[h!]
\centering
\includegraphics[scale=\Mysize]{toggle}
\caption{Exemplo de configuração de um toggle para envio e recebimento de mensagens}
\end{figure}

Neste capítulo vamos verificar como enviar e receber mensagens.
Para maiores informações quanto a este tipo de mensagens, consulte o código dos
objetos send e receive no repositório do PD\footnote{
\url{https://github.com/libpd/libpd/blob/master/pure-data/src/x_connective.c}
}.

% -----+-----+-----+-----+-----+-----+-----+-----+-----+-----+-----+-----+-----+
%      |     |     |     |     |     |     |     |     |     |     |     |     |
% -----+-----+-----+-----+-----+-----+-----+-----+-----+-----+-----+-----+-----+
\section{Enviando mensagens}

O envio de mensagem pode ser feito por meio de algumas funções:

\begin{itemize}
\item \texttt{pd\_bang(t\_pd *x)}
\item \texttt{pd\_float(t\_pd *x, t\_float f)}
\item \texttt{pd\_symbol(t\_pd *x, t\_symbol *s)}
\item \texttt{pd\_pointer(t\_pd *x, t\_gpointer *gp)}
\item \texttt{pd\_list(t\_pd *x, t\_symbol *s, int argc, t\_atom *argv)}
\item \texttt{typedmess(t\_pd *x, t\_symbol *s, int argc, t\_atom *argv)}
\end{itemize}

A última função, que não define tipo para a mensagem, é uma espécie de ``anything''.
O primeiro parâmetro para todas as funções pode ser conseguido em um símbolo
em seu atributo \texttt{s\_thing}.

No exemplo mailman.c, o objeto envia mensagens para um conjunto de diferentes 
seletores recebidos como parâmetro.
As mensagens são do tipo bang e enviadas quando o objeto recebe um bang.

\begin{lstlisting}[caption=Envio de mensagens bang por send]
void mailman_bang(t_mailman *x){
   int i = 0;
   for(; i < x->argc ; i++){
      if (x->messages[i]->s_thing)
         pd_bang(x->messages[i]->s_thing);
   }
}
\end{lstlisting}

O resultado pode ser visto na Figura \ref{fig:mailman}.
\begin{figure}[h!]
\centering
\includegraphics[scale=\Mysize]{mailman}
\caption{Envio de mensagens por send}
\end{figure}


% -----+-----+-----+-----+-----+-----+-----+-----+-----+-----+-----+-----+-----+
%      |     |     |     |     |     |     |     |     |     |     |     |     |
% -----+-----+-----+-----+-----+-----+-----+-----+-----+-----+-----+-----+-----+
\section{Receive}

O recebimento de mensagens junto ao PD ocorre por meio de uma função
\texttt{bind} com o o símbolo esperado.

\begin{itemize}
   \item \texttt{pd\_bind(t\_pd *x, t\_symbol *s)}
   \item \texttt{pd\_unbind(t\_pd *x, t\_symbol *s)}
\end{itemize}

Nestas funções, o primeiro parâmetro é seu objeto e o segundo parâmetro é a
mensagem que vc espera receber.

\begin{lstlisting}[caption=Exemplo de objeto que recebe várias mensagens]
void * postbox_new(t_symbol *s, int argc, t_atom *argv){
   t_postbox *x = (t_postbox *) pd_new(postbox_class);
   int i = 0;
   int counter = 0;
   for(; i < argc ; i++){
      if((argv + i)->a_type == A_SYMBOL)
         counter++;
   }
   x->messages = getbytes(counter * sizeof(t_symbol *));
   counter = 0;
   for(i = 0; i < argc ; i++){
      if((argv + i)->a_type == A_SYMBOL){
         x->messages[counter] = atom_getsymbol(argv + i);
         pd_bind(&x->x_obj.ob_pd, x->messages[counter]);
         counter++;
      }
   }
//   pd_bind(&x->x_obj.ob_pd, gensym("#key"));
//   pd_bind(&x->x_obj.ob_pd, gensym("#keyname"));
//   pd_bind(&x->x_obj.ob_pd, gensym("#keyup"));
   x->argc = counter;
   x->x_outlet = outlet_new(&x->x_obj, gensym("bang"));
   return (void *) x;
}
\end{lstlisting}

É importante que, no destrutor do objeto, a ligação seja desfeita para evitar
que o PD aborte ao tentar enviar uma mensagem para um objeto que não existe mais.
Isto é feito pela função \texttt{unbind}, apresentada abaixo.

\begin{lstlisting}[caption=Desvinculando o objeto com a mensagem no destrutor]
void postbox_destroy(t_postbox *x) {
   outlet_free(x->x_outlet);
   int i = 0;
   for(; i < x->argc ; i++){
      pd_unbind(&x->x_obj.ob_pd, x->messages[i]);
   }
//   pd_unbind(&x->x_obj.ob_pd, gensym("#key"));
//   pd_unbind(&x->x_obj.ob_pd, gensym("#keyname"));
//   pd_unbind(&x->x_obj.ob_pd, gensym("#keyup"));
}
\end{lstlisting}


Uma vez associado o recebimento de um determinado símbolo, é necessário definir
qual método será chamado para cada tipo de mensagem recebida.
A definição destes métodos é similar a definição dos inlets ativos.
É ideal que um objeto que possua um receive possua métodos para receber todos
os tipos de mensagens do PD, mesmo que tais métodos não sejam implementados.

\begin{lstlisting}[caption=Associando métodos para receber mensagens]
void postbox_list_method(t_postbox *x, t_symbol *s, int argc, t_atom *argv){
   post("list %s", atom_getsymbolarg(1, argc, argv)->s_name);
}

void postbox_setup(void) {
   postbox_class = class_new(gensym("postbox"),
      (t_newmethod) postbox_new, // Constructor
      (t_method) postbox_destroy, // Destructor
      sizeof (t_postbox),
      CLASS_NOINLET,
      A_GIMME,
      0);//Must always ends with a zero

   class_addbang(postbox_class, postbox_bang_method);
   class_addfloat(postbox_class, postbox_float_method);
   class_addlist(postbox_class, postbox_list_method);
}
\end{lstlisting}

Note que este objeto (postobox.c) não possui inlets e por isto tais métodos
só serão usados para mensagens.
Caso o mesmo possua inlets, o tratamento de mensagens recebidas pelo inlet ou
pelo receive será exatamente o mesmo, o que é muito bacana.


% -----+-----+-----+-----+-----+-----+-----+-----+-----+-----+-----+-----+-----+
%      |     |     |     |     |     |     |     |     |     |     |     |     |
% -----+-----+-----+-----+-----+-----+-----+-----+-----+-----+-----+-----+-----+
\section{Indo além disto}

Entender como receber mensagens enviadas pelo PD ajudará a entender como trocar
mensagens entre um objeto PD em C e sua interface em tcl/tk.
Além disto, é possível receber e enviar mensagens padrões do PD, como movimentos
de teclado, entradas MIDI e assim por diante.

Com isto, é possível desenvolver um \external que envia eventos de teclado ou
que recebe eventos de teclado diretamente.

Alguns exemplos de mensagens internas do PD que são trocadas por send / receive
são:

\begin{itemize}
   \item Eventos de teclado\footnote{Retirados de: \url{https://github.com/libpd/libpd/blob/master/pure-data/src/x_gui.c}}
   \begin{itemize}
      \item \texttt{\#key}
      \item \texttt{\#keyup}
      \item \texttt{\#keyname}
   \end{itemize}
   \item Eventos MIDI\footnote{Retirados de: \url{https://github.com/libpd/libpd/blob/master/pure-data/src/x_midi.c}}
   \begin{itemize}
      \item \texttt{\#midiin}
      \item \texttt{\#sysexin}
      \item \texttt{\#notein}
      \item \texttt{\#ctlin}
      \item \texttt{\#pgmin}
      \item \texttt{\#bendin}
      \item \texttt{\#touchin}
      \item \texttt{\#polytouchin}
      \item \texttt{\#midiclkin}
      \item \texttt{\#midirealtimein}
      \item \texttt{\#midiclkin}
      \item \texttt{\#midiclkin}
   \end{itemize}
\end{itemize}

\todo{Acho que seria bacana alguns exemplos disto tudo...}

\chapter{Clock}

Muitos eventos, quando tratamos de música, são dependentes do tempo.
O Pure Data, segundo o autor, é perfeitamente sincronizado no tempo podendo,
obviamente, ter problemas com o tempo real já que o mundo real não é perfeitamente
sincronizado.
Para sincronizar eventos discretos, o PD utiliza, internamente, seu relógio DSP.
Isto significa que, todos os eventos são agendados para ocorrer em um determinado
bloco de amostras podendo o usuário ajustar a precisão destes eventos alterando
o tamanho do bloco de processamento ou a taxa de amostragem.
Para evitar que cada \external utilize um relógio próprio para agendar eventos
discretos, o PD conta com funções específicas para tratar o tempo.



\part{GUI}
\chapter{Manipulando GUI}

O PD foi desenvolvido utilizando um modelo que separa a apresentação das
funcionalidades.
Isto rola com 2 linguagens de programação: C para a engine e tcl/tk para as GUI.
GUI e engine se comunicam por um socket que permite, inclusive, que a engine
do PD seja executada em uma máquina e a GUI em outra.

% -----+-----+-----+-----+-----+-----+-----+-----+-----+-----+-----+-----+-----+
%      |     |     |     |     |     |     |     |     |     |     |     |     |
% -----+-----+-----+-----+-----+-----+-----+-----+-----+-----+-----+-----+-----+
\section{Iniciando no Tcl/Tk}
O Tcl (Tool Command Language) é uma linguagem de programação dinâmica bastante poderosa e simples de ser utilizada. 
Tk é um conjunto de ferramentas para construção de GUI de aplicações desktop e é a GUI padrão não apenas do TCL mas 
de várias outras linguagens e pode ser executada nativamente em vários sistemas operacionais modernos como Windows,
 Mac OS X, Linux, entre outros \footnote{Visite: http://www.tcl.tk/ para maiores informações}.

O Tk possui vários objetos prontos para GUI como botões, labels, janelas, checkbox, entre outros. Os objetos criados
devem ser armazenados em uma variável. Toda variável em Tk possui um nome que inicia com ponto (.). Após criar o objeto
e definir seus atributos, basta que o mesmo seja empacotado (pack).

\begin{lstlisting}
label .hello -text "Hello World"
pack .hello
\end{lstlisting}

Uma vez que um programa tk esteja pronto, basta salvá-lo com a extensão .tcl e utilizar um interpretador para executá-lo.
Um exemplo deste interpretador no Linux é o wish. Assim, salvando o exemplo anterior com o nome de helloWorld.tcl e executando

\begin{lstlisting}
wish helloWorld.tcl
\end{lstlisting}

Teremos o resultado:
\begin{figure}[ht!]
	\centering
	\includegraphics[width=0.7\textwidth]{helloWorld}
	\caption{Hello World no Tcl/Tk}
\end{figure}

No diretório tk deste tutorial temos exemplos mais interessantes de GUI com Tk mas a prática desta linguagem
vai além do escopo deste tutorial. Um tutorial mais completo de Tcl/Tk pode ser encontrado em 
\footnote{http://www.bin-co.com/tcl/tutorial/} e uma lista dos objetos e parâmetros pode ser encontrada em
\footnote{http://www.tkdocs.com/widgets/index.html}.



% ----------------------------------------------------------------------------
% GUI
% ----------------------------------------------------------------------------

\chapter{Externals com GUI - Usando o Tk/Tcl}
External GUIs - Using Tk/Tcl (Look the tk folder)
--------------------------------------------------------------------------
Example 14 - My First external with GUI
\begin{figure}[h!]
	\centering
	\includegraphics[width=0.7\textwidth]{example14}
	\caption{Adicionando GUI tk.}
\end{figure}

Example 15 - Adding GUI components


% ----------------------------------------------------------------------------
% MISCELÂNEAS
% ----------------------------------------------------------------------------

\chapter{Miscelâneas}
Como saber se um objeto existe?
pd\_findbyclass

Exemplo: http://pure-data.svn.sourceforge.net/viewvc/pure-data/trunk/pd/src/d\_array.c?revision=10432\&view=markup 

Linha 82


post

error


\part{Usando o PD para construir aplicações}
\chapter{De \externals para código C}

Como transformar seu patch em código C?

Vamos ver!

\end{document}

