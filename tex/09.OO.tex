% ----------------------------------------------------------------------------
% ORIENTAÇÃO A OBJETOS
% ----------------------------------------------------------------------------

\chapter{Orientação a Objetos}

Parte do texto foi retirada de http://www.katjaas.nl/pitchshift/soundtouch~.html.

Este cara é um exemplo de external que utiliza uma biblioteca OO:
http://pure-data.svn.sourceforge.net/viewvc/pure-data/trunk/externals/fftw/fftw~.c?revision=6017&view=markup
No caso, a bibliteca utilizada é a FFTW3. Outro exemplo é o PDCUDA do Drebs!

Outro exemplo que utilizad OO em external é o pixopencv.

http://pure-data.svn.sourceforge.net/viewvc/pure-data/trunk/externals/pix_opencv/

É possível criar externals utilizando C++ e orientação a objetos. A parte crítica de tal implementação é garantir que o PD consiga enxergar as funções dentro do objeto compilado do C++. É necessário pensar nisto não apenas como uma alternativa para criar externals em C++ mas também para utilizar bibliotecas C++ para a criação de externals. Utilizar estas bibliotecas torna necessário que a mesma seja compilada na forma como símbolos em C. Para garantir isto é necessário utilizar a definição extern "C".

TERMINAR ESTE EXEMPLO. COMO FAZER PARA O MAKEFILE FUNCIONAR TAMBÉM PARA ARQUIVOS .H e .CPP?


\begin{lstlisting}
extern "C" void example18_setup(void) {
    example18_class = class_new(gensym("example18"),
            (t_newmethod) example18_new, // Constructor
            0,
            sizeof (t_example18),
	    CLASS_NOINLET,
            0);
};
\end{lstlisting}

Usando o setup "externalizado" o mesmo passa a ser exportado e compilado em um objeto com esta função "visível". Deve ser o suficiente para compilar o external com sua API utilizando C++. Para compilar, usados um compilador C++ como o g++ do Linux.

\begin{lstlisting}
LINUXCFLAGS = -msse -DPD -DUNIX -DICECAST -O3 -funroll-loops -fomit-frame-pointer -fcheck-new \
    -Wall -W -Wshadow \
    -Wno-unused -Wno-parentheses -Wno-switch -fvisibility=hidden

LINUXINCLUDE =  -I ./include

	g++ $(LINUXCFLAGS) $(LINUXINCLUDE) -c *.cpp 
	g++ --export-dynamic -shared -o $*.pd_linux *.o -lc -lm -lstdc++ 
	strip --strip-unneeded $(NAME).pd_linux
	rm -f *.o ../$(NAME).pd_linux
\end{lstlisting}


