% ----------------------------------------------------------------------------
% OS TIPOS DE DADOS DO PD
% ----------------------------------------------------------------------------
\chapter{Os tipos de dados do PD}

Uma vez que o Pure Data é utilizado em diversas plataformas, muitos tipos
comuns de variáveis, como \texttt{int}, são redefinidos. Para escrever um
\external que seja portável para qualquer plataforma, é razoável que você
utilize os tipos providos pelo Pure Data. Como dissemos na seção
\ref{sec:organizacao}, para escrever um \external, é necessário incluir o
arquivo \texttt{m\_pd.h} que possui definições de constantes (versão do Pure
Data, sistema operacional, compilador, etc), estruturas, assinaturas de
funções e tipos de dados.

Existem muitos tipos predefinidos que devem fazer a vida do programador mais
simples. Em geral, os tipos do pd têm nome iniciado por \texttt{t\_}.

\begin{center}
\begin{tabular}{|l|l|}
\hline
tipo do pd & descrição \\
\hline
\texttt{t\_atom}      & átomo \\
\texttt{t\_float}     & valor de ponto flutuante \\
\texttt{t\_symbol}    & símbolo \\
\texttt{t\_gpointer}  & ponteiro (para objetos gráficos) \\
\texttt{t\_int}       & valor inteiro \\
\texttt{t\_signal}    & estrutura de um sinal \\
\texttt{t\_sample}    & valor de um sinal de áudio (ponto flutuante) \\
\texttt{t\_outlet}    & \emph{outlet} de um objeto \\
\texttt{t\_inlet}     & \emph{inlet} de um objeto \\
\texttt{t\_object}    & objeto gráfico \\
\texttt{t\_class}     & uma classe do pd \\
\texttt{t\_method}    & um método de uma classe \\
\texttt{t\_newmethod} & ponteiro para um construtor (uma função \texttt{\_new}) \\
\hline
\end{tabular}
\end{center}

% -----+-----+-----+-----+-----+-----+-----+-----+-----+-----+-----+-----+-----+
%      |     |     |     |     |     |     |     |     |     |     |     |     |
% -----+-----+-----+-----+-----+-----+-----+-----+-----+-----+-----+-----+-----+
\section{Símbolos}

Um símbolo corresponde a um valor constante de uma \emph{string}, ou seja,
uma sequência de letras que formam uma palavra única.

Cada símbolo é armazenado em uma tabela de busca por razões de performance. A
função \texttt{gensym(char *)} procura por uma string em uma tabela de busca e
retorna o endereço daquele símbolo. Se a string não foi encontrada na tabela,
um novo símbolo é adicionado.

Estes símbolos serão usados para várias coisas como para criar e associar mensagens 
entre objetos, definir ações esperadas para mensagens recebidas por inlets, criar 
comunicação entre a GUI e o Pd, entre outras.

Para imprimir, por exemplo, o valor de uma String contida em um \texttt{t\_symbol} é
necessário acessar sua propriedade \texttt{s\_name}.

Exemplo:

\begin{lstlisting}[caption=Imprimindo o texto de um símbolo]
   printf("%s\n", my_symbol->s_name);
\end{lstlisting}

% -----+-----+-----+-----+-----+-----+-----+-----+-----+-----+-----+-----+-----+
%      |     |     |     |     |     |     |     |     |     |     |     |     |
% -----+-----+-----+-----+-----+-----+-----+-----+-----+-----+-----+-----+-----+
\section{Mensagens}
\label{sec:mensagens}

Dados que não correspondem a áudio são distribuídos via um sistema de
mensagens.
Cada mensagem é composta de um ``seletor" e uma lista de átomos.

\subsection{Átomos}
\label{sec:atomos}

Um átomo é um tipo de dado do PD que possui um valor e uma identificação.
Os tipos de átomo mais utilizados são:

\begin{itemize}
\item \texttt{A\_FLOAT}: um valor numérico (de ponto flutuante).
\item \texttt{A\_SYMBOL}: um valor simbólico (string).
\item \texttt{A\_POINTER}: um ponteiro.
\end{itemize}


Valores numéricos são sempre considerados valores de ponto flutuante
(\texttt{t\_float}), mesmo que possam ser exibidos como valores inteiros.

Átomos do tipo \texttt{A\_POINTER} não são muito importantes (para \externals
simples).

O tipo de um átomo \texttt{a} é armazenado no elemento da estrutura
\texttt{a.a\_type}.

Outros tipos de átomo definidos no arquivo \texttt{m\_pd.h} são:

\begin{itemize}
\item \texttt{A\_NULL},
\item \texttt{A\_FLOAT},
\item \texttt{A\_SYMBOL},
\item \texttt{A\_POINTER},
\item \texttt{A\_SEMI},
\item \texttt{A\_COMMA},
\item \texttt{A\_DEFFLOAT},
\item \texttt{A\_DEFSYM},
\item \texttt{A\_DOLLAR},
\item \texttt{A\_DOLLSYM},
\item \texttt{A\_GIMME},
\item \texttt{A\_CANT}
\end{itemize}

Nem todos estes átomos são utilizados no desenvolvimento de \externals e alguns
são representações internas do PD para símbolos reservados como vírgula, cifrão
ou nulo.
Assim, se um objeto precisa passar para outro objeto um valor nulo, um cifrão ou
vírgula, estes tipos devem ser utilizado.

A manipulação de átomos pode ser feita pelas seguintes funções:

\subsubsection{\texttt{SETFLOAT}}
\texttt{SETFLOAT(atom, f)}

Esta macro define o tipo do átomo como \texttt{A\_FLOAT} e armazena no mesmo o valor de f.
É necessário passar um ponteiro para o átomo.

\subsubsection{\texttt{SETSYMBOL}}
\texttt{SETSYMBOL(atom, s)}

Esta macro define o tipo do átomo como \texttt{A\_SYMBOL} e armazena no mesmo um ponteiro
para o símbolo s.
É necessário passar um ponteiro para o átomo.

\subsubsection{\texttt{SETPOINTER}}
\texttt{SETPOINTER(atom, pt)}

Esta macro define o tipo do átomo como \texttt{A\_POINTER} e armazena no mesmo o ponteiro
pt.
É necessário passar um ponteiro para o átomo.

\subsubsection{\texttt{atom\_getfloat}}
\texttt{t\_float atom\_getfloat(t\_atom *a)}

Se o átomo for do tipo \texttt{A\_FLOAT}, retorna o valor do float, caso contrário,
retorna 0.0.

\subsubsection{\texttt{atom\_getfloatarg}}
\texttt{t\_float atom\_getfloatarg(int which, int argc, t\_atom *argv)}

Se o tipo do átomo encontrado na lista de átomos argv, de tamanho argc e na
posição which for \texttt{A\_FLOAT}, retorna o valor deste átomo.
Caso contrário, retorna 0.0.

\subsubsection{\texttt{atom\_getint}}
\texttt{t\_int atom\_getint(t\_atom *a)}

Se o átomo for do tipo \texttt{A\_INT}, retorna o valor inteiro, caso contrário,
retorna 0.

\subsubsection{\texttt{atom\_getintarg}}
\texttt{t\_int atom\_getintarg(int which, int argc, t\_atom *argv)}

Se o tipo do átomo encontrado na lista de átomos argv, de tamanho argc e na
posição which for \texttt{A\_INT}, retorna o valor deste átomo.
Caso contrário, retorna 0.

\subsubsection{\texttt{atom\_getsymbol}}
\texttt{t\_symbol atom\_getsymbol(t\_atom *a)}

Se o átomo for do tipo \texttt{A\_SYMBOL}, retorna um ponteiro para este símbolo, caso
contrário, retorna ``0''.

\subsubsection{\texttt{atom\_getsymbolarg}}
\texttt{t\_symbol atom\_getsymbolarg(int which, int argc, t\_atom *argv)}

Se o tipo do átomo encontrado na lista de átomos argv, de tamanho argc e na
posição which for \texttt{A\_SYMBOL}, retorna um ponteiro para este símbolo.
Caso contrário, retorna ``0''.

\subsubsection{\texttt{atom\_gensym}}
\texttt{t\_symbol *atom\_gensym(t\_atom *a)}

Se o átomo for do tipo \texttt{A\_SYMBOL}, retorna um ponteiro para este símbolo.
Átomos de outros tipos são convertidos de maneira ``razoável'' em string,
adicionados na tabela de símbolos, e um ponteiro para este símbolo é retornado.

\subsubsection{\texttt{atom\_string}}
\texttt{void atom\_string(t\_atom *a, char *buf, unsigned int bufsize)}

Converte um átomo em uma string (char *) previamente alocada e de tamanho bufsize.

\subsection{Seletores}

Um seletor é um símbolo que define o tipo de uma mensagem. Existe cinco
seletores pré-definidos:

\begin{itemize}
  \item \texttt{bang}: rotula um gatilho de evento. Uma mensagem de
    \texttt{bang} consiste somente do seletor e não contém uma lista de átomos.
  \item \texttt{float} rotula um valor numérico. A lista de uma mensagem
    \texttt{float} contém um único átomo de tipo \texttt{A\_FLOAT}.
  \item \texttt{symbol} rotula um valor simbólico. A lista de uma mensagem
    \texttt{symbol} consiste em um único átimo do tipo \texttt{A\_SYMBOL}.
  \item \texttt{pointer} rotula um valor de ponteiro. A lista de uma mensagem do
    tipo \texttt{pointer} contém um único átimo do tipo \texttt{A\_POINTER}.
  \item \texttt{list} rotula uma lista de um ou mais átomos de tipos arbitrários.
\end{itemize}

Uma vez que os símbolos para estes seletores são utilizados com frequência,
seu endereço na tabela de símbolos pode ser utilizado diretamente, sem a
necessidade da utilização de \texttt{gensym}:

\begin{center}
\begin{tabular}{|l|l|l|}
\hline
  seletor & rotina de busca & endereço de busca \\
\hline
  \texttt{bang} & \texttt{gensym("bang")} & \texttt{\&s\_bang} \\
  \texttt{float} & \texttt{gensym("float")} & \texttt{\&s\_float} \\
  \texttt{symbol} & \texttt{gensym("symbol")} & \texttt{\&s\_symbol} \\
  \texttt{pointer} & \texttt{gensym("pointer")} & \texttt{\&s\_pointer} \\
  \texttt{list} & \texttt{gensym("list")} & \texttt{\&s\_list} \\
  \texttt{--}  (signal) & \texttt{gensym("signal")} & \texttt{\&s\_signal} \\
\hline
\end{tabular}
\end{center}

Outros seletores também podem ser utilizados. A classe receptora tem que
prover um médodo para um seletor específico ou para \texttt{anything}, que
corresponde a qualquer seletor arbitrário.

Mensagens que não possuem seletor explícito e começam com um valor numérico
são reconhecidas automaticamente como mensagens \texttt{float} (se consistirem
de apenas um átomo) ou como mensagens \texttt{list} (se forem compostas de
diversos átomos).

Por exemplo, as mensagens \texttt{12.429} e \texttt{float 12.429} são
idênticas. Da mesma forma, as mensagens \texttt{list 1 para voce} é idêntica a
\texttt{1 para voce}.

\todo{Cabe escrever sobre binbuf?}
