% ----------------------------------------------------------------------------
% OS TIPOS DE DADOS DO PD
% ----------------------------------------------------------------------------

\chapter{Os tipos de dados do PD}

Uma vez que o Pure Data é utilizado em diversas plataformas, muitos tipos
comuns de variáveis, como \texttt{int}, são redefinidos. Para escrever um
\external que seja portável para qualquer plataforma, é razoável que você
utilize os tipos providos pelo Pure Data. Como dissemos na seção
\ref{sec:organizacao}, para escrever um \external, é necessário incluir o
arquivo \texttt{m\_pd.h} que possui definições de constantes (versão do Pure
Data, sistema operacional, compilador, etc), estruturas, assinaturas de
funções e tipos de dados.

Existem muitos tipos predefinidos que devem fazer a vida do programador mais
simples. Em geral, os tipos do pd têm nome iniciado por \texttt{t\_}.

\begin{center}
\begin{tabular}{|l|l|}
\hline
pd-type & description \\
\hline
\texttt{t\_atom} & \texttt{atom} \\
\texttt{t\_float} & \texttt{floatingpoint value} \\
\texttt{t\_symbol} & \texttt{symbol} \\
\texttt{t\_gpointer} & \texttt{pointer (to graphical objects)} \\
\texttt{t\_int} & \texttt{integer value} \\
\texttt{t\_signal} & \texttt{structure of a signal} \\
\texttt{t\_sample} & \texttt{audiosignal-value (floatingpoint)} \\
\texttt{t\_outlet} & \texttt{outletof an object} \\
\texttt{t\_inlet} & \texttt{inlet of an object} \\
\texttt{t\_object} & \texttt{object-interna} \\
\texttt{t\_class} & \texttt{a pd-class} \\
\texttt{t\_method} & \texttt{class-method} \\
\texttt{t\_newmethod} & \texttt{pointer to a constructor (new-routine)} \\
\hline
\end{tabular}
\end{center}

\section{Símbolos}

Um símbolo corresponde a um valor constante de uma \emph{string}, ou seja, a
uma sequência de letras que formam uma palavra.

Cada símbolo é armazenado em uma tabela de busca por razões de performance. A
função \texttt{gensym(char *)} procura por uma string em uma tabela de busca e
retorna o endereço daquele símbolo. Se a string não foi encontrada na tabela,
um novo símbolo é adicionado.


\texttt{t\_symbol} e \texttt{*gensym}

\section{Mensagens}

Dados que não correspondem a áudio são distribuídos via um sistema de
mensagens. Cada mensagem consiste em um ``seletor" e uma lista de átomos.

\subsection{Átomos}

Existem três tipos de átomos:

\begin{itemize}
\item \texttt{A\_FLOAT}: um valor numérico (de ponto flutuante).
\item \texttt{A\_SYMBOL}: um valor simbólico (string).
\item \texttt{A\_POINTER}: um ponteiro.
\end{itemize}

Valores numéricos são sempre considerados valores de ponto flutuante
(\texttt{t\_float}), mesmo que possam ser exibidos como valores inteiros.

Átomos do tipo \texttt{A\_POINTER} não são muito importantes (para \externals
simples).
Since pd is used on several plattforms, many ordinary types of variables, like int, are re-defined. To write portable code, it is reasonable to use types provided by pd.

O tipo de um átomo \texttt{a} é armazenado no elemento da estrutura
\texttt{a.a\_type}.

\subsection{Seletores}

Um seletor é um símbolo que define o tipo de uma mensagem. Existe cinco
seletores pré-definidos:

\begin{itemize}
  \item \texttt{bang}: rotula um gatilho de evento. Uma mensagem de
    \texttt{bang} consiste somente do seletor e não contém uma lista de átomos.
  \item \texttt{float} rotula um valor numérico. A lista de uma mensagem
    \texttt{float} contém um único átomo de tipo \texttt{A\_FLOAT}.
  \item \texttt{symbol} rotula um valor simbólico. A lista de uma mensagem
    \texttt{symbol} consiste em um único átimo do tipo \texttt{A\_SYMBOL}.
  \item \texttt{pointer} rotula um valor de ponteiro. A lista de uma mensagem do
    tipo \texttt{pointer} contém um único átimo do tipo \texttt{A\_POINTER}.
  \item \texttt{list} rotula uma lista de um ou mais átomos de tipos arbitrários.
\end{itemize}

Uma vez que os símbolos para estes seletores são utilizados com frequência,
seu endereço na tabela de símbolos pode ser utilizado diretamente, sem a
necessidade da utilização de \texttt{gensym}:

\begin{center}
\begin{tabular}{|l|l|l|}
\hline
  selector & lookup-routine & lookup-address \\
\hline
  \texttt{bang} & \texttt{gensym("bang")} & \texttt{\&s\_bang} \\
  \texttt{float} & \texttt{gensym("float")} & \texttt{\&s\_float} \\
  \texttt{symbol} & \texttt{gensym("symbol")} & \texttt{\&s\_symbol} \\
  \texttt{pointer} & \texttt{gensym("pointer")} & \texttt{\&s\_pointer} \\
  \texttt{list} & \texttt{gensym("list")} & \texttt{\&s\_list} \\
  \texttt{--}  (signal) & \texttt{gensym("signal")} & \texttt{\&s\_symbol} \\
\hline
\end{tabular}
\end{center}

Outros seletores também podem ser utilizados. A classe receptora tem que
prover um médodo para um seletor específico ou para \texttt{anything}, que
corresponde a qualquer seletor arbitrário.

Mensagens que não possuem seletor explícito e começam com um valor numérico
são reconhecidas automaticamente como mensagens \texttt{float} (se consistirem
de apenas um átomo) ou como mensagens \texttt{list} (se forem compostas de
diversos átomos).

Por exemplo, as mensagens \texttt{12.429} e \texttt{float 12.429} são
idênticas. Da mesma forma, as mensagens \texttt{list 1 para voce} é idêntica a
\texttt{1 para voce}.

