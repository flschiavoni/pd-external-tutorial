% ----------------------------------------------------------------------------
% Introdução
% ----------------------------------------------------------------------------
 
\chapter{Introdução}

Pure Data, ou simplesmente Pd, é um ambiente visual de programação musical que
permite a criação de aplicações musicais complexas a partir da combinação de
componentes visuais mais simples chamados \textbf{objetos}.
As distribuições oficiais do Pure Data contêm diversos objetos prontos para o
uso, mas também permitem a extensão de suas funcionalidades através da criação
de novos objetos utilizando o próprio Pure Data, Tcl/Tk (plugins gráficos) e C/C++.
Desta forma, novas linhas de código escritas pelo usuário são compilados como
bibliotecas dinâmicas e podem ser carregadas pelo programa em tempo de
execução.
Objetos desta forma levam o nome de \textbf{\externals}.

Este é um tutorial prático para o desenvolvimento de \externals para o
Pure Data.
A iniciativa de escrever este documento surgiu no primeiro semestre
de 2011, durante a disciplina de Computação Musical ministrada pelo professor
Marcelo Gomes de Queiroz no Instituto de Matemática e Estatística da
Universidade de São Paulo.
A intenção deste tutorial é auxiliar programadores a desenvolver \externals de
maneira bastante simples através de exemplos práticos.

Mais do que ampliar a gama de objetos do Pure Data e criar novos objetos, o
objetivo deste trabalho é também fornecer ao pesquisador de computação musical
uma ferramenta para implementar e testar algoritmos de processamento de áudio
para caráter de estudo.
Isto significa que podemos reimplementar várias coisas que já existem no Pure
Data simplesmente porque é didático programar e colocar algoritmos para
funcionar.

Não é objetivo deste tutorial ensinar processamento de som, ensinar algoritmos, 
ensinar programar (C, C++, TCL/tk) ou ensinar a utilizar o Pure Data.
Também não é objetivo questionar o modo como o Pd e seus externals foram feitos.

\todo{Este tutorial ainda não está pronto e por isto você encontrará caixinhas
como esta com notas do que mais temos de fazer.}

É importante dizer que nada no mundo se aprende sozinho.
Foi graças aos vários \externals escritos para o Pd, com seu código aberto e
documentado que conseguimos reunir o conhecimento que aqui presente.
Seria impossível citar todos os autores  de \externals que nos ajudaram sem
saber.
No entanto, não deixamos de agradecer ao que chamamos de comunidade de software
livre, ao autor do Pd (seria Public Domain?),  Miller Puckette e ao autor de
outro tutorial, IOHannes Zmölnig.
Muito obrigado.

Este tutorial está acompanhado de vários exemplos cujos códigos ilustram os nossos
passos.

Este tutorial é dividido em 3 partes: \externals feitos no próprio ambiente Pure
Data, feitos em Tcl/Tk e em C.

% -----+-----+-----+-----+-----+-----+-----+-----+-----+-----+-----+-----+-----+
%      |     |     |     |     |     |     |     |     |     |     |     |     |
% -----+-----+-----+-----+-----+-----+-----+-----+-----+-----+-----+-----+-----+
\section{Sobre \Externals}
O Pure Data possui uma distribuição do autor, chamada Vanilla.
Nela, Miller Puckette possui internals e externals desenvolvidos em C, Tcl/tk e
no próprio Pd.
Os objetos desenvolvidos no Pd são comumente chamados também de abstrações.
Independentemente de como um objeto é feito, o termo \external é utilizado para
tudo o que não está no Vanilla, ou seja, tudo que é externo à distribuição
oficial do autor.

% -----+-----+-----+-----+-----+-----+-----+-----+-----+-----+-----+-----+-----+
%      |     |     |     |     |     |     |     |     |     |     |     |     |
% -----+-----+-----+-----+-----+-----+-----+-----+-----+-----+-----+-----+-----+
\section{Arquivos de ajuda}

É importante distribuir, junto com novos \externals, um arquivo de ajuda do
Pure Data com instruções e exemplos de utilização. Como convenção, o arquivo
de ajuda deve ter o mesmo nome que o \external, acrescido do sufixo
\courier{-help.pd}. Por exemplo, para o código fonte \courier{example01.c}, que
gera o objeto \courier{example01.pd\_linux}, escrevemos também o arquivo
\courier{example01-help.pd}.


No capítulo \ref{sec:basico} veremos uma forma de alterar o nome do arquivo de
ajuda em \externals feitos em C com a opção de ajuda que aparece no menu
contextual com um clique do botão direito no objeto do \external dentro do Pure Data.

% -----+-----+-----+-----+-----+-----+-----+-----+-----+-----+-----+-----+-----+
%      |     |     |     |     |     |     |     |     |     |     |     |     |
% -----+-----+-----+-----+-----+-----+-----+-----+-----+-----+-----+-----+-----+
\section{Utilizando \externals}
\label{sec:using}

Para que um novo \external possa ser utilizado no Pure Data, é necessário
instalá-lo em um caminho que o Pure Data possa encontrá-lo
\footnote{Retirado do site \url{http://puredata.info/docs/faq/how-do-i-install-externals-and-help-files}}
.
Há algumas pastas padrões para instalações destes objetos para que os
mesmos possam ser instalados no Pure Data de maneira separada do programa em si.
Isto permite que o Pure Data seja atualizado sem interferir na instalação destes
novos \externals.
Tais diretórios servem para instalar bibliotecas, \externals, classes de objetos,
abstrações, pluguins GUI e arquivos de ajuda.

\textbf{GNU/Linux}
\begin{itemize}
   \item Package manager \courier{/usr/lib/pd/extra}
   \item Apenas para o usuário \courier{~/pd-externals}
   \item Global \courier{/usr/local/lib/pd-externals}
\end{itemize}

\textbf{Mac OS X}
\begin{itemize}
   \item Apenas para o usuário \courier{~/Library/Pd}
   \item Global \courier{/Library/Pd} (É necessário criar esta pasta)
\end{itemize}

\textbf{Windows}
\begin{itemize}
   \item Apenas para o usuário \courier{\%AppData\%\char`\\Pd}
   \item Global \courier{\%CommonProgramFiles\%\char`\\Pd}
\end{itemize}

\nota{Nota para sistemas Windows:}{

O diretório do Windows \courier{\%AppData\%\char`\\Pd} deverá ser algo como \\
\courier{C:\char`\\Documents and Settings\char`\\myusername\char`\\Application Data\char`\\Pd} \\
que também é sinônimo de \\
\courier{\%UserProfile\%\char`\\Application Data\char`\\Pd} \\

O diretório \courier{\%UserProfile\%} é seu diretório raiz, que em sistemas
em inglês normalmente é chamado de \\
\courier{C:\char`\\Documents and Settings\char`\\myusername}.

Já o diretório \courier{\%CommonProgramFiles\%\char`\\Pd} em um windows em inglês
normalmente se refere ao diretório \\
\courier{C:\char`\\Program Files\char`\\Common Files\char`\\Pd}\\
( en español: \courier{C:\char`\\Archivos de programa\char`\\Archivos comunes\char`\\Pd}, auf Deutsch: 
\courier{C:\char`\\Programme\char`\\Gemeinsame Dateien\char`\\Pd}).
Este diretório é sinônimo de \courier{\%ProgramFiles\%\char`\\Common Files\char`\\Pd}.
}

É possível ainda adicionar o diretório que contém o arquivo binário do \external ao
caminho de busca do Pure Data, de forma que para acessá-lo de dentro de um
\patch não seja necessário digitar o caminho inteiro até o objeto.
Isto pode ser feito através da passagem de um parâmetro na linha de comando do Pure Data
com a opção \courier{-path <caminho>}, ou de forma gráfica acessando a opção
\courier{File} $\rightarrow$ \courier{Path...} no menu do Pure Data,
como pode ser visto na figura \ref{fig:path}.

\grafico{path}{Adicionando o diretório de um \external ao caminho de busca do Pure Data.}

Para carregar uma biblioteca de \externals (mais de um \external no mesmo
arquivo-fonte), é possível indicar o nome da bibliotecai na linha
de comando do Pure Data utilizando a opção \courier{-lib <biblioteca>}, ou
também graficamente através do menu \courier{File} $\rightarrow$
\courier{Startup...}, como pode ser visto na figura \ref{fig:startup}.

\grafico{startup}{Adicionando uma biblioteca ao Pure Data.}

Uma vez que o novo \external está no caminho de busca do Pure Data, é possível
carregá-lo em seu \patch.
Para carregar um \external em um \courier{patch} do Pure Data em tempo de
execução, basta criar um objeto (com \courier{CTRL+1} ou acessando o menu
\courier{Put} $\rightarrow$ \courier{Object}) com o caminho (relativo ou
absoluto) para o objeto compilado com a biblioteca compartilhada, omitindo a
extensão.

% -----+-----+-----+-----+-----+-----+-----+-----+-----+-----+-----+-----+-----+
%      |     |     |     |     |     |     |     |     |     |     |     |     |
% -----+-----+-----+-----+-----+-----+-----+-----+-----+-----+-----+-----+-----+
\section{Exemplos}
Este tutorial é acompanhado de diversos exemplos de \externals que ilustram o
conteúdo coberto pelo mesmo.
Vale lembrar que muitos destes objetos são de utilidade duvidosa e servem apenas
como exemplos didáticos.
Os arquivos de exemplo estão no mesmo repositório que este material.
No início de cada capítulo há uma caixa apresentando quais exemplos podem ser
utilizados.
