% ----------------------------------------------------------------------------
% CONSTRUTOR E DESTRUTOR
% ----------------------------------------------------------------------------

\chapter{Construtor e destrutor}

O Construtor de um objeto pode receber parâmetros. Estes parâmetros são
ilustrados abaixo.

\begin{figure}[h!]
	\centering
	\includegraphics[width=0.7\textwidth]{example2}
	\caption{External recebendo parâmetros. Note a tela de saída no fundo da imagem.}
\end{figure}

\section{Construtor}

Parâmetros de inicialização no construtor podem permitir que inicializemos o
external com determinados valores. Isto é feito definindo os parâmetros no
métodos class\_new() quanto na definição da função construtora. (Veja o
exemplo02).

\begin{lstlisting}
// Constructos of the class
void * example2_new(t_symbol * arg1, t_floatarg arg2) {
    t_example2 *x = (t_example2 *) pd_new(example2_class);
    post("First arg: %s", arg1->s_name);
    post("Second arg: %f", arg2);
    return (void *) x;
}

void example2_setup(void) {
    example2_class = class_new(gensym("example2"),
            (t_newmethod) example2_new, // Constructor
            0,
            sizeof (t_example2),
	    CLASS_NOINLET,
            A_DEFFLOAT, // First Constructor parameter
            A_DEFSYMBOL, // Second Constructor parameter
            0);
}
\end{lstlisting}

\todo{Estranho que no método class\_new, o float foi definido primeiro.}

Notem que os parâmetros são definidos com um tipo e são recebidos com outro.
Como explicado na seção \ref{sec:mensagens}, todos os dados que não
correspondem a sinais de áudio são transmitidos como mensagens, compostas de
átomos. Para ver os tipos de átomo que podem ser utilizados na passagem de
parâmetros, veja a seção \ref{sec:atomos}.

Para aceitar qualquer tipo de átomo na passagem de um parâmetro específico,
utilize o tipo de átomo \texttt{A\_GIMME} (veja o exemplo09).

\begin{lstlisting}
// Constructor of the class
void * example9_new(t_symbol *s, int argc, t_atom * argv) {
   t_example9 *x = (t_example9 *) pd_new(example9_class);
   post("%d parameters received",argc);
   return (void *) x;
}

void example9_setup(void) {
   example9_class = class_new(gensym("example9"),
     (t_newmethod) example9_new, // Constructor
     (t_method) example9_destroy, // Destructor
     sizeof (t_example9),
     CLASS_NOINLET,
     A_GIMME, // Allows various parameters
     0); // LAST argument is ALWAYS zero
}
\end{lstlisting}

Quando utilizamos o tipo de átomo \texttt{A\_GIMME} o método construtor
funciona como uma função \texttt{main()} em C: ela recebe os parâmetros
\texttt{argc}, que indica o número de átomos na lista, e \texttt{*argv}, que
aponta para a lista de átomos de fato. Veja o exemplo na figura
\ref{fig:construtor-parametros}.

\begin{figure}[h!]
\centering
\includegraphics[width=0.7\textwidth]{example9}
\caption{Diferente da linguagem C, o primeiro parâmetro não é o nome do external.}
\label{fig:construtor-parametros}
\end{figure}

Note que o Pure Data não obriga que o usuário passe parâmetros para o objeto. É
como se todo construtor, independentemente de como ele está definido, aceitasse
sua instanciação vazia. Cabe ao programador verificar se os parâmetros
recebidos são em quantidade, tipo e valor esperado e, caso não seja, abortar a
construção do objeto e não retornar sua instância.

\section{Destrutor}

O destrutor de uma classe permite liberar alguma memória eventualmente alocada
pelo construtor ou por outras funções do \external (veja o exemplo 07).

\begin{lstlisting}
// Destroy the object
void example9_destroy(t_example9 *x) {
  post("You say good bye and I say hello");
}

void example9_setup(void) {
   example9_class = class_new(gensym("example9"),
     (t_newmethod) example9_new, // Constructor
     (t_method) example9_destroy, // Destructor
     sizeof (t_example9),
     CLASS_NOINLET,
     A_GIMME, // Allows various parameters
     0); // LAST argument is ALWAYS zero
}
\end{lstlisting}

A liberação da memória pode ser feita utilizando a função \texttt{freebytes()}
definida na API do Pure Data.

\begin{lstlisting}
void freebytes(void *x, size_t nbytes)
\end{lstlisting}

