\chapter{Plugins Tcl}

\example{
\begin{itemize}
\item \texttt{helloworld-plugin.tcl}
\item \texttt{menucolor-plugin.tcl}
\end{itemize}
}

O PD foi desenvolvido utilizando um modelo que separa a apresentação das
funcionalidades.
Para isto, utiliza 2 linguagens de programação: C para a engine e TCL/TK para as
GUI.
GUI e engine se comunicam por um socket que permite, inclusive, que a engine
do PD seja executada em uma máquina e a GUI em outra.

O Tcl (Tool Command Language) é uma linguagem de programação dinâmica bastante
poderosa e simples de ser utilizada. 
Tk é um conjunto de ferramentas para construção de GUI de aplicações desktop e é
a GUI padrão não apenas do TCL mas de várias outras linguagens e pode ser
executada nativamente em vários sistemas operacionais modernos como Windows, Mac
OS X, Linux, entre outros \footnote{Visite: http://www.tcl.tk/ para maiores
informações}.

\nota{Para conhecer Tcl/Tk}{
Para aprender Tcl/Tk indicamos um tutorial que pode ser encontrado em:\\
\url{http://www.bin-co.com/tcl/tutorial/} \\
Uma lista dos objetos e parâmetros pode ser encontrada em\\
\url{http://www.tkdocs.com/widgets/index.html}
}

Um plugin TCL costuma ser visto como um plugin de GUI que permite alterar a
aparência e funcionalidades da interface gráfica do PD.
Um plugin de GUI é um arquivo escrito em tcl (de extensão \courier{.tcl}) e que
\textbf{obrigatoriamente} tem o nome \courier{-plugin.tcl}.
Ou seja, para escrever um plugin de nome \courier{teste}, o mesmo deverá estar em
um arquivo de nome.

Para testarmos o funcionamento dos comandos tcl no PD, é possível utilizarmos o
TCL Prompt.
No Pure Data Vanila o mesmo se encontra no Menu Help $>$ TCL Prompt, conforme
apresentado na fig.\ref{fig:tcl_prompt}.

\grafico{tcl_prompt}{Prompt Tcl do Pure Data.}

Exemplos de plugins de GUI e documentações sobre os mesmos podem ser encontrados em:
\begin{itemize}
   \item \url{https://svn.code.sf.net/p/pure-data/svn/branches/pd-gui-rewrite/0.43/startup/}
   \item \url{http://puredata.info/docs/guiplugins/GuiPluginsAPI}
   \item \url{http://puredata.info/docs/guiplugins/SimpleExamples}
   \item \url{http://puredata.info/docs/guiplugins/GUIPlugins}
\end{itemize}

% -----+-----+-----+-----+-----+-----+-----+-----+-----+-----+-----+-----+-----+
%      |     |     |     |     |     |     |     |     |     |     |     |     |
% -----+-----+-----+-----+-----+-----+-----+-----+-----+-----+-----+-----+-----+
\section{Iniciando no Tcl/Tk - Hello World}

O Pure Data fornece diversas funções TCL que podem ser utilizadas para
manipular sua GUI.
Tais funções permitem o acesso a todos os widgets de GUI do Pure Data.
O primeiro exemplo plugin escreve textos no console, conforme comandos
apresentados no Código \ref{code:console}\footnote{Exemplo retirado de \url{http://puredata.info/docs/guiplugins/GuiPluginsAPI}}.

\begin{lstlisting}[caption={Escrevendo no console},label={code:console}]
::pdwindow::verbose 1 "Hello, World!\n"
::pdwindow::verbose 0 "Hello, World!\n"
::pdwindow::error "Houston, we have a problem!\n"
::pdwindow::fatal "See you on the other side.\n"
::pdwindow::post  "This message will self destruct in five seconds.\n"
::pdwindow::debug "Second phase initiated\n"
\end{lstlisting}

Tal código apresenta um exemplo de como utilizar a API TCL do Pure Data para
alterar sua interface com o usuário.
O resultado deste código pode ser visto na fig.\ref{fig:console}.

\grafico{console}{Escrevendo no console}

% -----+-----+-----+-----+-----+-----+-----+-----+-----+-----+-----+-----+-----+
%      |     |     |     |     |     |     |     |     |     |     |     |     |
% -----+-----+-----+-----+-----+-----+-----+-----+-----+-----+-----+-----+-----+
\section{Conhecendo a GUI do PD}

Para alterar os elementos de GUI do PD é necessário saber o nome destes objetos.
Vale lembrar quem Tcl/Tk, os componentes gráficos são armazenados em uma variável
de maneira hierárquica e que todo nome de variável inicia com um ponto (\courier{.}).
A variável \courier{.} (apenas ponto) remete a aplicação principal.
O Código \ref{code:listaprincipal} apresenta um comando que lista no terminal
todos os filhos da aplicação principal.

\begin{lstlisting}[caption=Executando o Hello World,label={code:listaprincipal}]
foreach c [winfo children .] \
{::pdwindow::post "[winfo name $c]\n"}
\end{lstlisting}

O resultado deste comando pode ser conferido na fig.\ref{fig:pdwindow}.

\grafico{pdwindow}{Lista de widgets da janela do PD.}

Com isto podemos verificar, por exemplo, que a janela principal do PD chama-se
\courier{.pdwindow} e que a barra de menus chama-se \courier{.menubar}.
Podemos utilizar um comando similar ao Código \ref{code:listaprincipal} para
listar todos os elementos da janela principal do PD, conforme apresentado no
Código \ref{code:listaPd}.

\begin{lstlisting}[caption={Comando TCL para listar os elementos da janela principal},
label={code:listaPd}]
foreach c [winfo children .pdwindow] \
{::pdwindow::post "[winfo name $c]\n"}
\end{lstlisting}

A saída deste comando lista os seguintes objetos:
\begin{itemize}
   \item menubar  -barra de menus
   \item header   - Painel superior, onde está liga DSP
   \item tcl      - Painel inferior onde estou executando os comandos TCL
   \item text     - Área do Log
   \item scroll   - barras de rolagem da área de log
\end{itemize}

Podemos continuar examinando recursivamente o nome de todos os elementos
gráficos que compõem a GUI do PD, como os menus
(Código \ref{code:listaMenu}), as janelas de \patch e assim por diante.

\begin{lstlisting}[caption={Comando TCL para listar os elementos do menu},label={code:listaMenu}]
foreach c [winfo children .menubar] \
 {::pdwindow::post "[winfo name $c]\n"}

Output:
   file
   edit
   put
   find
   media
   window
   help
\end{lstlisting}

Com isto podemos conhecer recursivamente todos os elementos que compõe a GUI
do PD.
O Codigo \ref{code:listaHeader} apresenta os elementos do cabeçalho 
(\courier{.pdwindow.header}).

\begin{lstlisting}[caption={Comando TCL para listar os elementos do cabeçalho},label={code:listaHeader}]
foreach c [winfo children .pdwindow.header] \
 {::pdwindow::post "[winfo name $c]\n"}

Output:
   pad1
   dsp
   ioframe
   loglabel
   logmenu
   exitButton
\end{lstlisting}

Apesar de ser possível encontrar na documentação do PD e no código-fonte o nome
de todos os componentes, esta seção pretendeu possibilitar a exploração destes
componentes.

% -----+-----+-----+-----+-----+-----+-----+-----+-----+-----+-----+-----+-----+
%      |     |     |     |     |     |     |     |     |     |     |     |     |
% -----+-----+-----+-----+-----+-----+-----+-----+-----+-----+-----+-----+-----+
\section{Alterando a GUI do PD}

Uma vez que sabemos o nome dos elementos gráficos do Pd, podemos agora
alterá-los e reconfigurá-los.
O Código \ref{code:alteraMenu} apresenta exemplos que alteram as propriedades
dos objetos menu.

\begin{lstlisting}[caption={Exemplo de alteração de menus do PD com Tcl (plugin exemplo \courier{menucolor-plugin.tcl}).},
label={code:alteraMenu}]
.menubar configure  -foreground red
.menubar configure  -background black
\end{lstlisting}

O resultado de algumas alterações de configuração de itens de GUI do PD podem ser
vistos na Fig.\ref{fig:menubar}.

\grafico{menubar}{Alterando a cor da barra de menus}

Outro exemplo seria alterar a cor de fundo do texto ou do cabeçalho,
como no código ~\ref{code:mudacortexto}

\begin{lstlisting}[caption={Alteração da cor do texto de log e do cabeçalho.},label={code:mudacortexto}]
.pdwindow.text configure -background green
.pdwindow.header configure -background green
\end{lstlisting}


% -----+-----+-----+-----+-----+-----+-----+-----+-----+-----+-----+-----+-----+
%      |     |     |     |     |     |     |     |     |     |     |     |     |
% -----+-----+-----+-----+-----+-----+-----+-----+-----+-----+-----+-----+-----+
\section{Variáveis Globais}

Além dos componentes (widgets) gráficos, há ainda na programação TCL do PD uma
série de variáveis globais que podem ser utilizadas pelo programador.
Tais variáveis guardam parâmetros de inicialização e de execução da GUI do PD
como, por exemplo, a versão, o sistema de janela (que permite identificar se
o PD está sendo rodado em Linux / Windows / Mac), a fonte padrão do PD, entre
outras.

O código \ref{code:variavelglobal}, retirado dos arquivos fontes do PD (\courier{pd-gui.tcl}),
lista algumas destas variáveis e seus valores no momento de inicialização do PD.

\begin{lstlisting}[caption={Variáveis globais.},label={code:variavelglobal}]
PD_MAJOR_VERSION 0
PD_MINOR_VERSION 0
PD_BUGFIX_VERSION 0
PD_TEST_VERSION ""
done_init 0

TCL_MAJOR_VERSION 0
TCL_MINOR_VERSION 0
TCL_BUGFIX_VERSION 0

# for testing which platform we are running on ("aqua", "win32", or "x11")
windowingsystem ""

# args about how much and where to log
loglevel 2
stderr 0

# connection between 'pd' and 'pd-gui'
host ""
port 0

# canvas font, received from pd in pdtk_pd_startup, set in s_main.c
font_family "courier"
font_weight "normal"

# sizes of chars for each of the Pd fixed font sizes:
#  fontsize  width(pixels)  height(pixels)
set font_fixed_metrics {
    8 5 11
    9 6 12
    10 6 13
    12 7 16
    14 8 17
    16 10 19
    18 11 22
    24 14 29
    30 18 37
    36 22 44
}
font_measured_metrics {}

# root path to lib of Pd's files, see s_main.c for more info
sys_libdir {}

# root path where the pd-gui.tcl GUI script is located
sys_guidir {}

# user-specified search path for objects, help, fonts, etc.
sys_searchpath {}

# hard-coded search patch for objects, help, plugins, etc.
sys_staticpath {}

# the path to the folder where the current plugin is being loaded from
current_plugin_loadpath {}

# list of command line flags set at startup
startup_flags {}

# list of libraries loaded on startup
startup_libraries {}

# start dirs for new files and open panels
filenewdir [pwd]
fileopendir [pwd]

# lists of audio/midi devices and APIs for prefs dialogs
audio_apilist {}
audio_indevlist {}
audio_outdevlist {}
midi_apilist {}
midi_indevlist {}
midi_outdevlist {}
pd_whichapi 0
pd_whichmidiapi 0

# current state of the DSP
dsp 0

# state of the peak meters in the Pd window
meters 0

# the toplevel window that currently is on top and has focus
focused_window .

# store that last 10 files that were opened
recentfiles_list {}
total_recentfiles 10

# keep track of the location of popup menu for PatchWindows, in canvas coords
popup_xcanvas 0
popup_ycanvas 0

# modifier for key commands (Ctrl/Control on most platforms, Cmd/Mod1 on MacOSX)
modifier ""

# current state of the Edit Mode menu item
editmode_button 0

# variables for holding the menubar to allow for configuration by plugins
::pdwindow_menubar ".menubar"
::patch_menubar   ".menubar"
::dialog_menubar   ""

# minimum size of the canvas window of a patch
canvas_minwidth 50
canvas_minheight 20

# undo states
::undo_action "no"
::redo_action "no"
::undo_toplevel "."

\end{lstlisting}

Tais variáveis serão utilizadas em alguns exemplos deste tutorial.

% -----+-----+-----+-----+-----+-----+-----+-----+-----+-----+-----+-----+-----+
%      |     |     |     |     |     |     |     |     |     |     |     |     |
% -----+-----+-----+-----+-----+-----+-----+-----+-----+-----+-----+-----+-----+
\section{Estilo de código}

O arquivo fonte do PD \courier{pd-gui.tcl} traz algumas considerações quanto
ao estilo de código em TCL.
Conforme tal arquivo, estas ideias são preliminares e poderão mudar com o tempo.

\begin{itemize}
   \item Quando possível, utilizar aspas duplas para delimitar mensagens.
   \item Utilize \courier{\$::myvar} em vez de \courier{global myvar}.
   \item Por questões de clareza, evite códigos ``soltos''.
   Todo código deverá estar em uma função (\courier{proc}) que é disparada pela
   função \courier{main()}.
   \item Se um procedimento \courier{menu\_*} abre um painel de diálogo, este
   procedimento deve ser chamado \courier{menu\_*\_dialog}.
   \item Utilize \courier{eq/ne} para comparações de \courier{string} e não
   \courier{== / !=}\footnote{Retirado de \url{http://wiki.tcl.tk/15323}}.
\end{itemize}

Nome para variáveis comuns:

\begin{itemize}
   \item \courier{\$window}     = Qualquer tipo de widget Tk que pode ser uma janela
   \item \courier{\$mytoplevel} = A identificação de uma janela feita por um comando \courier{toplevel}
   \item \courier{\$gfxstub}    = Um id de janela de diálogo ``toplevel`` feita por \courier{gfxstub/x\_gui.c}
   \item \courier{\$menubar}    = O ``menu'' pertencente a cada \courier{toplevel}
   \item \courier{\$mymenu}     = O ``menu'' pertencente a uma barra de menus, como o menu Arquivo
   \item \courier{\$tkcanvas}   = O ``canvas'' Tk que é a raiz de todos os \patches
\end{itemize}

Tipos de painéis de diálogo.

\begin{itemize}
   \item global(Há apenas um):   find, sendmessage, prefs, helpbrowser
   \item por canvas:          font, canvas properties (Criados com uma mensagem do Pd)
   \item por objeto:          gatom, iemgui, array, data structures (Criados com uma mensagem do Pd)
\end{itemize}

\begin{comment}
O Pure Data permite que vários textos da sua GUI sejam internacionalizados, ou seja,
traduzidos dependendo do idioma do sistema operacional.
Para garantir que um texto seja traduzido utilizamos o mesmo 

\begin{lstlisting}
[_ "About Pd"]
\end{lstlisting}
\end{comment}
