\chapter{Introdução}

\example{
\begin{itemize}
\item \texttt{oscillator.pd}
\item \texttt{additive.pd}
\item \texttt{sum.pd}
\item \texttt{sum-help.pd}
\item \texttt{gain\til.pd}
\item \texttt{additive\til.pd}
\item \texttt{volume\til.pd}
\item \texttt{poliosc.pd}
\end{itemize}
}

A reutilização de código é uma técnica da computação que permite o uso de um
software existente para o desenvolvimento de um novo software.
Com isto, mais do que reaproveitar o código escrito, é possível reaproveitar uma
solução existente.
A possibilidade de pensar em cada parte do código individualmente permite ainda
dividir o trabalho a ser feito e pensar se cada código é a melhor solução para
resolver um determinado problema.

O Pure Data permite a reutilização dos \patches desenvolvidos em outros projetos
trazendo assim um novo nível de abstração para programadores do ambiente.
Esta reutilização permite ainda programar o PD de maneira modular ou mesmo
distribuída no caso de uma criação coletiva, por exemplo, onde cada pessoa pode
ficar responsável por criar uma parte do \patch.

Tais códigos reutilizáveis do PD são chamados de \externals e permitem que um
usuário estenda o ambiente com seus próprios objetos, encapsulando em novas
abstrações trechos de código que podem ser reutilizados.

% -----+-----+-----+-----+-----+-----+-----+-----+-----+-----+-----+-----+-----+
%      |     |     |     |     |     |     |     |     |     |     |     |     |
% -----+-----+-----+-----+-----+-----+-----+-----+-----+-----+-----+-----+-----+
\section{Meu primeiro \external}

A criação de um \external usando o próprio Pure Data depende de criar um \patch
e salvá-lo com um determinado nome.
Caso este \patch seja colocado na mesma pasta de um novo \patch ou na pasta raiz
do Pure Data, tal objeto já pode ser reusado em outros \patches.

\grafico{oscillator}{Meu primeiro \external em Pure Data.}

Note na fig.\ref{fig:oscillator} que o \external foi salvo utilizando o nome
``oscillator''.

\grafico{oscillator2}{Utilizando meu primeiro \external em Pure Data.}

A partir deste momento, outros \patches podem utilizá-lo chamando-o pelo nome
(\textbf{oscillator} \obj{oscillator}), como apresentado na Fig.\ref{fig:oscillator2}

% -----+-----+-----+-----+-----+-----+-----+-----+-----+-----+-----+-----+-----+
%      |     |     |     |     |     |     |     |     |     |     |     |     |
% -----+-----+-----+-----+-----+-----+-----+-----+-----+-----+-----+-----+-----+
\section{Passando parâmetros para seu \external}

Muitas vezes, para o reaproveitamento do código, é necessário a configuração do
\external de maneira a permitir que o mesmo seja instanciado com parâmetros
diferentes.
Imagine, por exemplo, o caso de um filtro que pode ter sua frequência de corte
ajustada a cada utilização.
Uma maneira de configurar um \external escrito em Pd é a passagem de parâmetros
para o objeto.
Sabemos que o valor \$0 permite uma identificação única para o \patch.
O Pd permite receber os parâmetros de inicialização de um objeto seguindo a
mesma numeração onde o primeiro parâmetro é o \$1, o
segundo parâmetro é o \$2 e assim por diante, conforme ilustrado na
Fig.\ref{fig:additive}.

\grafico{additive}{Recebendo parâmetros em um \external.}

A Fig.\ref{fig:additive2} apresenta este objeto sendo inicializado com
parâmetros.

\grafico{additive2}{Passando parâmetros para o \external.}

% -----+-----+-----+-----+-----+-----+-----+-----+-----+-----+-----+-----+-----+
%      |     |     |     |     |     |     |     |     |     |     |     |     |
% -----+-----+-----+-----+-----+-----+-----+-----+-----+-----+-----+-----+-----+
\section{IOLets}

Um \external criado em Pd pode ainda possuir inlets e outlets, para receber
dados de controle, e inlet\til  e outlet\til, para receber áudio.
Para utilizá-los é necessário utilizar os objetos \obj{inlet}, \obj{outlet},
\obj{inlet\til} e \obj{outlet\til} do Pd.
A figura \ref{fig:sum} apresenta um externo que recebe dados.

\grafico{sum}{Adicionando iolets de controle a um \external.}

A Fig.\ref{fig:sum2} apresenta este objeto com seus respectivos iolets.

\grafico{sum2}{Utilizando os iolets de um \external.}

O mesmo serve para os iolets de áudio, conforme apresentado na
Fig.\ref{fig:gain} e Fig.\ref{fig:gain2}.

\grafico{gain}{Adicionando iolets de áudio a um \external.}

É importante notar que os iolets seguem a ordem de cima para baixo, da esquerda
para a direita, ou seja, alterar a posição dos objetos iolets em um \external
pode alterar a ordem do recebimento de mensagens.

\grafico{gain2}{Utilizando os iolets de áudio de um \external.}

Desta maneira, é possível configurar um \external tanto com os parâmetros de
inicialização quanto durante a execução do código, conforme apresentado na
fig.\ref{fig:additivetil} e fig.\ref{fig:additivetil2}.

\grafico{additivetil}{Utilizando inlets, outlets e parâmetros.}

\grafico{additivetil2}{Arquivo de help do exemplo anterior.}

% -----+-----+-----+-----+-----+-----+-----+-----+-----+-----+-----+-----+-----+
%      |     |     |     |     |     |     |     |     |     |     |     |     |
% -----+-----+-----+-----+-----+-----+-----+-----+-----+-----+-----+-----+-----+
\section{Externals com interface gráfica}

É possível ainda que um objeto exporte sua interface gráfica para o patch que
o utiliza.
Isto ocorre alterando a propriedade do objeto (clicando com o botão contrário
sobre o mesmo e selecionando no menu contextual a opção ``properties'').
Na caixa de diálogo de propriedades aparecerá a opção ``graph on parent'',
conforme ilustrado pela fig.\ref{fig:volume1}.

\grafico{volume1}{Janela de diálogo de preferências de um \external.}

Note que é possível ainda definir se o objeto apresentará ou não seu nome e o
tamanho de sua interface gráfica.

Uma vez que tal opção for selecionada, aparecerá no objeto uma caixa vermelha
que indica a região do objeto que será utilizada como interface gráfica,
conforme apresentado na fig.\ref{fig:volume}.

\grafico{volume}{Definição da área visível do \external.}

Todos os objetos que estão inteiramente dentro desta área vermelha serão
apresentados como interface gráfica do objeto no \patch onde o mesmo for
utilizado, como ilustra a fig.\ref{fig:volume2}.

\grafico{volume2}{Apresentação do \external no \patch.}

% -----+-----+-----+-----+-----+-----+-----+-----+-----+-----+-----+-----+-----+
%      |     |     |     |     |     |     |     |     |     |     |     |     |
% -----+-----+-----+-----+-----+-----+-----+-----+-----+-----+-----+-----+-----+
\section{Externals dinâmicos}

O Pure data permite que um \patch altere seu próprio conteúdo ou o conteúdo de
outros \patches em tempo de execução por meio do envio de determinadas
mensagens. 
Tal técnica, também conhecida por programação dinâmica, possibilita que um
\external possua, por exemplo, uma quantidade de objetos que variam de acordo
com seu parâmetro inicial.
Apesar de bastante trabalhosa, esta técnica permite muito mais reuso dos objetos
criados, conforme ilustra a fig.\ref{fig:poliosc2}

\grafico{poliosc2}{Utilizando \externals dinâmicos.}

A criação dinâmica, apresentada na fig.\ref{fig:poliosc}, utiliza o envio de
mensagens para criação de objetos e para a conexão de objetos.
Este exemplo cria dinamicamente osciladores e conecta-os à saída de som.
A quantidade de osciladores e a frequência do primeiro oscilador é definida por
parâmetros e os demais osciladores terão frequências múltiplas do primeiro.

\grafico{poliosc}{Utilizando \externals dinâmicos.}

Caso a intenção seja criar iolets dinamicamente, haverá um problema de que todas
as conexões destes iolets serão perdidas quando o \patch principal é reaberto.
Isto ocorre pois no momento de carregar o \patch principal, os iolets ainda não
terão sido criados dinamicamente e por isto eles ainda não existem.

