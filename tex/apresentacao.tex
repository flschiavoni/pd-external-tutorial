% ----------------------------------------------------------------------------
% Introdução
% ----------------------------------------------------------------------------
 
\chapter{Introdução}

\example{
\begin{itemize}
\item \texttt{Makefile}
\end{itemize}
}


% -----+-----+-----+-----+-----+-----+-----+-----+-----+-----+-----+-----+-----+
%      |     |     |     |     |     |     |     |     |     |     |     |     |
% -----+-----+-----+-----+-----+-----+-----+-----+-----+-----+-----+-----+-----+
\section{Escrevendo \externals em C}

O código fonte do Pure Data é organizado de acordo com convenções de
programação orientada a objetos.
Para o desenvolvimento de \externals, é necessário seguir estas convenções e
fornecer ao ambiente uma nova classe com alguns métodos específicos, como
veremos mais adiante.
Para desenvolver para o Pure Data, é necessário importar o arquivo de cabeçalho
\texttt{m\_pd.h}\footnote{\url{http://pure-data.git.sourceforge.net/git/gitweb.cgi?p=pure-data/pure-data;a=blob\_plain;f=src/m\_pd.h;hb=HEAD}},
que contém definições de constantes, tipos e funções.

Uma boa fonte de informação é o tutorial de
\externals\footnote{\url{http://iem.at/pd/externals-HOWTO/pd-externals-HOWTO.pdf}}
escrito pelo IOHannes\footnote{\url{http://puredata.info/author/zmoelnig}}, um dos
programadores do Pure Data.
Apesar de ter utilizado este documento como ponto de partida, boa parte do que
está incluso no presente tutorial foi aprendido a partir da leitura do
código-fonte de \externals contidos no repositório oficial do Pure
Data\footnote{\url{http://pure-data.svn.sourceforge.net/viewvc/pure-data/trunk/externals/}}.

Navegando pelos códigos-fonte deste repositório você poderá notar que os programadores
que escreveram os externals que hoje estão disponível para o Pd seguiram estas convenções
e por isto a leitura destes códigos-fonte pode ser didática e simples.

Por esta razão, o primeiro conselho que damos para quem irá escrever \externals é seguir
estas convenções, mesmo que as mesmas não sejam a maneira como você está acostumado a 
programar deste jeito pois assim seu código também será didático e simples de entender.

Este tutorial não pretende cobrir os algoritmos de processamento de sinais mas explicar
como implementar estes algoritmos como objetos do Pd. Para processamento de sinais há
uma vasta bibliografia disponível que possui os algoritmos e o ferramental matemático
necessário para sua implementação.

% -----+-----+-----+-----+-----+-----+-----+-----+-----+-----+-----+-----+-----+
%      |     |     |     |     |     |     |     |     |     |     |     |     |
% -----+-----+-----+-----+-----+-----+-----+-----+-----+-----+-----+-----+-----+
\section{Organização do código-fonte e do objeto compilado}
\label{sec:organizacao}

Um novo \external corresponde a uma nova classe na arquitetura orientada a
objetos do Pure Data. Para que o carregamento da biblioteca dinâmica
em tempo de execução funcione corretamente, é necessário que o
arquivo binário produzido possua o mesmo nome que a classe correspondente ao
\external.

Para criar, por exemplo, um \external chamado ``passa-baixas'', podemos
escrever seu código-fonte em um arquivo chamado \texttt{passa-baixas.c}, e em
seguida compilar um objeto de biblioteca compartilhada chamado
\texttt{passa-baixas.pd\_linux}, no caso do sistema GNU/Linux.
Outras arquiteturas de sistema utilizam outras extensões para o nome do objeto com a
biblioteca compartilhada do \external, como por exemplo \texttt{.dll} (M\$
Windows), \texttt{.pd\_irix5} (SGI Irix) ou \texttt{.pd\_darwin} (Mac OS X).

\textbf{Importante:} O nome do arquivo com o código-fonte não possui formato
obrigatório, mas o nome do objeto compilado com a biblioteca dinâmica deve
sempre corresponder ao nome da classe, assim como sua extensão deve sempre
corresponder à arquitetura do sistema utilizado.

O mesmo cuidado é recomendado para os métodos que serão definidos internamente
no objeto. Os nomes de métodos que serão apresentados neste material seguem o padrão
encontrado no repositório do Pd.
É fortemente recomendado que o mesmo padrão seja utilizado em seu texto.

Para gerar a estrutura básica de um \external sugerimos utilizar o gerador de
\external disponível em \url{http://www.ime.usp.br/~fls/PDExternal-generator/PDExternal_generator.html}.

% -----+-----+-----+-----+-----+-----+-----+-----+-----+-----+-----+-----+-----+
%      |     |     |     |     |     |     |     |     |     |     |     |     |
% -----+-----+-----+-----+-----+-----+-----+-----+-----+-----+-----+-----+-----+
\section{Compilação}
\label{sec:compiling}

Para criar um objeto binário que pode ser carregado no Pure Data em tempo de
execução, primeiro compilamos o código fonte, criando assim um ou mais objetos
intermediários, e em seguida utilizamos um ligador (\emph{linker}) para criar
um objeto de biblioteca compartilhada.

No GNU/Linux, uma forma de realizar o processo
\texttt{example01.c} $\rightarrow$ \texttt{example01.o} $\rightarrow$
\texttt{example01.pd\_linux} é a seguinte:

\vspace{1em}
\begin{lstlisting}[caption=Compilação de um objeto]
EXTNAME=example01
cc -DPD -fPIC -Wall -o ${EXTNAME}.o -c ${EXTNAME}.c
ld -shared -lc -lm -o ${EXTNAME}.pd_linux ${EXTNAME}.o
rm ${EXTNAME}.o
\end{lstlisting}

A opção de compilação \texttt{-fPIC} resulta na criação de código binário que
roda independente de sua posição na memória, adequado para geração de
bibliotecas compartilhadas. A opção \texttt{-shared} passada para o ligador
determina a criação de uma biblioteca compartilhada.

Para facilitar a compilação, é interessante utilizar um \texttt{makefile}.
Os exemplos deste tutorial estão acompanhadas de um \texttt{makefile} encontrado
na seção de desenvolvedores do Pure
Data~\footnote{\url{http://puredata.info/docs/developer/MakefileTemplate}}.

Para compilar \externals no MacOS é necessário instalar o XCode.
Tem uma dezena de jeito de compilar pro Windows, usando o Mingw ou o C++ Builder.
Aqui\footnote{\url{http://puredata.hurleur.com/sujet-1029-problem-compiling-external-windows}}
temos exemplos e muitas discussões de como compilar externals no Windows.

\todo{Não testamos. Faltou coragem. Será que compensa testarmos isto tudo?}

\subsection{Debugando códigos}

Para verificar um erro, inicie o PD com seu patch de teste pelo terminal dentro do 
ambiente gdb com o comando

\begin{lstlisting}
gdb --args pd -path caminho-do-external

run
\end{lstlisting}
Caso o PD tenha algum problema em sua execução o GDB pode te ajudar a encontrá-lo.

Outros comandos básicos do gdb são \textbf{where} (que apresenta o arquivo e a linha onde
o erro ocorreu) e \textbf{list} (que mostra o código deste trecho).
Para navegar entre os arquivos, utilize \textbf{up} e \textbf{down}.
Para maiores informações, procure um tutorial sobre o gdb.

Além de debugar, pode ser útil verificar se o objeto está compilado corretamente.
Uma maneira de verificar isto é utilizar a ferramenta \texttt{nm} que lista os
símbolos de um objeto compilado.

\begin{lstlisting}
nm -D <external>.pd_linux
\end{lstlisting}

\subsection{Misturando código C e C++}

Existem algumas diferenças entre compiladores C e C++ que tornam a sintaxe das
linguagens incompatíveis, gerando resultados diferentes para um mesmo trecho
de código. Um exemplo disso que influencia o funcionamento de \externals no
Pure Data é a geração da tabela de símbolos dos objetos binários.

Compiladores C++ realizam um processo chamado \emph{name mangling} (ou
``dilaceramento de nomes''), que consiste em alterar o nome de funções,
estruturas, classes, etc, incluindo informações sobre o espaço de nomes do
objeto em questão. Isto resulta em nome diferentes gravados nas tabelas de
símbolos dos objetos binários, o que pode confundir o Pure Data no momento do
carregamento de um \external.

Para garantir que um compilador C++ gere nomes compatíveis com objetos
binários C, utilize a expressão \texttt{extern "C"} na frente dos nomes das
funções que serão chamadas pelo Pure Data:

\begin{lstlisting}[caption=Externalização de código C++]
extern "C" example01_setup(void);
extern "C" example01_new(void);
\end{lstlisting}



