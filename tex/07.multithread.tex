% ----------------------------------------------------------------------------
% MULTITHREADING
% ----------------------------------------------------------------------------

\chapter{Multithreading}

\example{
\begin{itemize}
\item \texttt{example20.c}
\end{itemize}
}

Como vimos no capítulo anterior, o bloco de processamento do Pd possui um
limite de tempo para a execução. É possível utilizar threads para separar
processos que consumam mais tempo do que o o período do bloco DSP, como por
exemplo no vaso de processos do tipo produtor/consumidor.

A programação multithread não é exatamente comum no Pure Data mas pode ser
útil para várias coisas como escrita em arquivo, envio de dados para a rede ou
atualização da interface gráfica (como veremos no próximo capítulo).

Apesar de existirem várias bibliotecas para programação paralela, como por
exemplo a simples utilização do comando fork do GNU/Linux, é desejável que os
externals do Pure Data sejam compatíveis com diversos sistemas operacionais.
Nos repositórios do Miller Puckette, autor do Pure Data, encontramos patches
nos quais ele utiliza threads POSIX implementadas pela biblioteca
\texttt{pthread} \footnote{Para maiores informações, visite:
https://computing.llnl.gov/tutorials/pthreads/}.

Note que esta solução, que em muito se aproxima da última forma de criar
inlets e outlets DSP implica em não trabalharmos mais em tempo real.
Implementações deste tipo não podem ser pensadas para processamentos aonde a
entrada de áudio será processada e devolvida na saída de áudio no mesmo bloco
de processamento do Pd.

\section{Criando threads}

Para utilizar a biblioteca de threads do POSIX é necessário incluir o arquivo
de cabeçalho correspondente. Em seguida, para armazenar as threads que
criamos, utilizamos uma variáveis que armazenam threads (veja o exemplo 20).

\begin{lstlisting}
#include <pthread.h>
...
typedef struct _example20 {
    t_object x_obj;
    pthread_t example20_thread;
} t_example20;

\end{lstlisting}

O próximo passo é criar uma função associada a esta thread e a enfim lançar a
thread. O lançamento da thread pode ser feito na função DSP. Isto implica
criar e iniciar uma thread nova em cada ciclo DSP do Pd.

\begin{lstlisting}
void * example20_thread_function(void * arg) {
  t_example20 *x = (t_example20 *) arg;
  while(1){
    // DO SOMETHING
    printf("Threading running!\n");
    sleep(1);
  }
  return 0;
}

static void example20_dsp(t_example20 *x, t_signal **sp){
  pthread_create(&x->example20_thread, NULL, example20_thread_function, x);
  dsp_add(example20_perform, 1 , x);
}
\end{lstlisting}

A função de criação da thread recebe o endereço da variável aonde a thread
será armazenada, os atributos da thread sendo criada\footnote{No caso,
passando \texttt{NULL} serão utilizados os atributos padrão. Para uma lista
completa dos atributos, visite:
http://sourceware.org/pthreads-win32/manual/pthread\_attr\_init.html}, a
função de inicialização associada a esta thread e os argumentos passados para
esta função. 

Caso seja passado mais de um argumento, é recomendado que se crie uma
estrutura de dados e que esta seja passada como argumento para a thread.

\section{Gerenciamento de threads}

Há diversas funções para o gerenciamento de threads, definidas no arquivo de
cabeçalho \texttt{pthread.h}. Entre elas:

\begin{itemize}
\item \texttt{pthread\_detach(threadid)}: Indica para a implementação que o
armazenamento da thread pode ser recuperado quando a mesma se encontra
terminada
\item \texttt{pthread\_join(threadid,status)}: Indica para o trecho de código
que chamou a thread que o mesmo deve esperar que a mesma tenha terminado sua
execução.
\item \texttt{pthread\_exit(void *value\_ptr)}: Encerra a execução de uma
thread e libera sua alocação de memória.
\end{itemize}

Em princípio, threads POSIX não possuem funções para interromper e continuar a
execução. Apesar disto, é possível implementar estes comandos por meio de
mutexes, como veremos a seguir.

\section{Controle de concorrência}

Uma das dificuldades de utilização de threads é o controle da concorrência por
recursos entre threads. Em situações de race condition, é necessário que
controlemos o acesso de threads concorrentes a trechos de código que acessam
dados comuns. Isto é feito por meio de mutex (mutual exclusion), sistemas de
controle atômicos que garantem que apenas uma thread será executada sobre um
trecho de código por vez.

Um mutex é definido da seguinte forma:

\begin{lstlisting}
pthread_mutex_t lock = PTHREAD_MUTEX_INITIALIZER;
pthread_cond_t cond = PTHREAD_COND_INITIALIZER;
int play = 0;
\end{lstlisting}

O controle ao trecho de código pode ser feito da seguinte forma:

\begin{lstlisting}
for(;;) { /* Playback loop */
  pthread_mutex_lock(&lock);
  while(!play) { /* We're paused */
    pthread_cond_wait(&cond, &lock); /* Wait for play signal */
  }
  pthread_mutex_unlock(&lock);
  /* Continue playback */
}
\end{lstlisting}

\todo{Creio que seria mais útil um exemplo funcional...}

\section{Controle via Pure Data}
Além de podermos trabalhar threads para nosso external, na biblioteca do
Pd há funções para que possamos criar mutex no Pd. São elas:

\todo{integrando multithread com o pd}

\begin{itemize}
\item \texttt{void sys\_lock(void);}
\item \texttt{void sys\_unlock(void);}
\item \texttt{int sys\_trylock(void);}
\end{itemize}
