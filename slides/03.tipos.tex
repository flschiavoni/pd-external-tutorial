\section{Tipos de dados}


\begin{frame}{Tipos definidos no arquivo \texttt{m\_pd.h}}
\begin{center}
\begin{tabular}{|l|l|}
\hline
tipo do pd & descrição \\
\hline
\texttt{t\_atom} & átomo \\
\texttt{t\_float} & valor de ponto flutuante \\
\texttt{t\_symbol} & símbolo \\
\texttt{t\_gpointer} & ponteiro (para objetos gráficos) \\
\texttt{t\_int} & valor inteiro \\
\texttt{t\_signal} & estrutura de um sinal \\
\texttt{t\_sample} & valor de um sinal de áudio (ponto flutuante) \\
\texttt{t\_outlet} & \emph{outlet} de um objeto \\
\texttt{t\_inlet} & \emph{inlet} de um objeto \\
\texttt{t\_object} & objeto gráfico \\
\texttt{t\_class} & uma classe do pd \\
\texttt{t\_method} & um método de uma classe \\
\texttt{t\_newmethod} & ponteiro para um construtor (uma função \texttt{\_new}) \\
\hline
\end{tabular}
\end{center}
\end{frame}


\begin{frame}{Símbolos}
Um símbolo é um valor constante de uma \emph{string}:
\begin{itemize}
\item Os símbolos são mantidos em uma tabela por razões de performance.
\item A função \texttt{gensym(char *)} faz a busca/criação de símbolos.
\end{itemize}
\end{frame}


\begin{frame}{Mensagens}
Uma mensagem é composta de um seletor e uma lista de átomos.

\vspace{1em}
Seletores:

\begin{center}
\begin{tabular}{|l|l|l|}
\hline
  seletor & rotina de busca & endereço de busca \\
\hline
  \texttt{bang} & \texttt{gensym("bang")} & \texttt{\&s\_bang} \\
  \texttt{float} & \texttt{gensym("float")} & \texttt{\&s\_float} \\
  \texttt{symbol} & \texttt{gensym("symbol")} & \texttt{\&s\_symbol} \\
  \texttt{pointer} & \texttt{gensym("pointer")} & \texttt{\&s\_pointer} \\
  \texttt{list} & \texttt{gensym("list")} & \texttt{\&s\_list} \\
  \texttt{--}  (signal) & \texttt{gensym("signal")} & \texttt{\&s\_signal} \\
\hline
\end{tabular}
\end{center}

Tipos de átomos:
\begin{itemize}
\item \texttt{A\_FLOAT}: um valor numérico (de ponto flutuante).
\item \texttt{A\_SYMBOL}: um valor simbólico (string).
\item \texttt{A\_POINTER}: um ponteiro.
\end{itemize}
\end{frame}


\begin{frame}{Outros tipos de átomo}
Existem diversos tipos de átomo definidos pelo arquivo \texttt{m\_pd.h} que
podem ser utilizados para passagem de parâmetros:
\begin{itemize}
\item \texttt{A\_NULL}
\item \texttt{A\_FLOAT}
\item \texttt{A\_SYMBOL}
\item \texttt{A\_POINTER}
\item \texttt{A\_SEMI}
\item \texttt{A\_COMMA}
\item \texttt{A\_DEFFLOAT}
\item \texttt{A\_DEFSYM}
\item \texttt{A\_DOLLAR} 
\item \texttt{A\_DOLLSYM}
\item \texttt{A\_GIMME}
\item \texttt{A\_CANT}
\end{itemize}
\end{frame}
