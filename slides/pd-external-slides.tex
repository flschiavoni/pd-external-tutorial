%-----------------------------------------------------------------------------
% cabeçalho
%-----------------------------------------------------------------------------
\documentclass{beamer}

% para criar um documento para acompanhamento, comente a linha acima e
% descomente as linhas abaixo.
%\documentclass[handout]{beamer}
%\usepackage{pgfpages}
%\pgfpagesuselayout{4 on 1}[a4paper,border shrink=5mm]


% localização
\usepackage[utf8x]{inputenc}
\usepackage[brazil]{babel}
%\usepackage{abntex}

% embelezamento
\usepackage{amssymb,amsmath,amsfonts}
\numberwithin{equation}{section}
\usepackage{enumerate}
\usepackage{xspace}
%\usetheme{Goettingen} % <---- so com crane
%\usetheme{sidebar}   % <---- com crane fica bom
%\usecolortheme{crane} % <---- Goettingen com crane
\usepackage{colortbl}
\usepackage{url}
\usepackage{color}
\usepackage{listings}             % Include the listings-package


% figuras
\usepackage{tikz}
\usetikzlibrary{arrows}

% Sumário no início de cada seção
\AtBeginSection[]
{
  \begin{frame}<beamer>{Estrutura da apresentação}
    \tableofcontents[currentsection]
  \end{frame}
}

% comandos
\definecolor{meuvermelho}{RGB}{200,40,40}
\definecolor{meuazul}{RGB}{40,40,140}
\newcommand{\enfase}[1]{{\color{meuvermelho}#1}}
\newcommand{\enfasel}[1]{{\color{meuazul}#1}}
\newcommand{\inlet}{\emph{inlet}}
\newcommand{\inlets}{\emph{inlets}}
\newcommand{\outlet}{\emph{outlet}}
\newcommand{\outlets}{\emph{outlets}}
\newcommand{\Inlet}{\emph{Inlet}}
\newcommand{\Inlets}{\emph{Inlets}}
\newcommand{\Outlet}{\emph{Outlet}}
\newcommand{\Outlets}{\emph{Outlets}}
\newcommand{\external}{\emph{external}\xspace}
\newcommand{\externals}{\emph{\enfase{externals}}\xspace}
\newcommand{\External}{\emph{External}\xspace}
\newcommand{\Externals}{\emph{Externals}\xspace}


%-----------------------------------------------------------------------------
% Dados para o primeiro slide
%-----------------------------------------------------------------------------

\title
{Como estender o Pure Data através de \emph{externals} em C}

%\subtitle
%{}

\author
{Flávio Luiz Schiavoni\\
\footnotesize{fls@ime.usp.br}}

\institute
[Universidade de São Paulo]
{
  Departamento de Ciência da Computação\\
  Instituto de Matemática e Estatística \\
  Universidade de São Paulo
}

\date{\today}


\lstset{ %
language=C,                % the language of the code
basicstyle=\footnotesize\ttfamily,       % the size of the fonts that are used for the code
numbers=left,                   % where to put the line-numbers
numberstyle=\tiny,      % the size of the fonts that are used for the line-numbers
%stepnumber=2,                   % the step between two line-numbers. If it's 1, each line 
                                % will be numbered
numbersep=8pt,                  % how far the line-numbers are from the code
backgroundcolor=\color{white},  % choose the background color. You must add \usepackage{color}
showspaces=false,               % show spaces adding particular underscores
showstringspaces=false,         % underline spaces within strings
showtabs=false,                 % show tabs within strings adding particular underscores
frame=single,                   % adds a frame around the code
tabsize=2,                      % sets default tabsize to 2 spaces
captionpos=b,                   % sets the caption-position to bottom
breaklines=true,                % sets automatic line breaking
breakatwhitespace=false,        % sets if automatic breaks should only happen at whitespace
title=\lstname,                 % show the filename of files included with \lstinputlisting;
                                % also try caption instead of title
escapeinside={\%*}{*)},         % if you want to add a comment within your code
%xleftmargin={2em},
%xrightmargin={2em},
columns=fixed,
morekeywords={*,...}            % if you want to add more keywords to the set
}


%-----------------------------------------------------------------------------
% Documento
%-----------------------------------------------------------------------------
\begin{document}

\begin{frame}
  \titlepage
\end{frame}

%-----------------------------------------------------------------------------
\begin{frame}{Objetivo da apresentação}

Introduzir à escrita de \emph{externals} para Pure Data, considerando:
\begin{itemize}
  \item Como escrever.
  \item Como compilar.
  \item Exemplos.
\end{itemize}
\end{frame}

%-----------------------------------------------------------------------------
\begin{frame}{Estrutura da apresentação}
  \tableofcontents
\end{frame}

% Since this a solution template for a generic talk, very little can
% be said about how it should be structured. However, the talk length
% of between 15min and 45min and the theme suggest that you stick to
% the following rules:  

% - Exactly two or three sections (other than the summary).
% - At *most* three subsections per section.
% - Talk about 30s to 2min per frame. So there should be between about
%   15 and 30 frames, all told.

%-----------------------------------------------------------------------------
% Seções
%-----------------------------------------------------------------------------
\section{Introdução}

\begin{frame}{Pure Data}
\begin{figure}
\centering
\includegraphics[width=0.7\textwidth]{../images/pd-facil}
\end{figure}
\end{frame}


\begin{frame}{Formas de estender o Pure Data}
Existem algumas formas de estender as funcionalidades do Pure Data:
\begin{itemize}
\item Subpatches.
\item \enfasel{Externals}.
\item Alterações no código-fonte (C e Tcl/Tk).
\end{itemize}
\pause
\vspace{1em}
Este seminário trata da extensão do Pure Data através da criação de
\externals em C.
\end{frame}


\begin{frame}[fragile]{Organização do código-fonte}
Algumas informações sobre o código do Pure Data:
\begin{itemize}
\item Publicado sob a licença \enfasel{Standard Improved BSD License}.
\item Organizado de forma orientada a objetos.
\begin{itemize}
\item Classes são tipos.
\item Objetos (gráficos) do Pure Data são instâncias de classes.
dados e assinaturas de funções: \texttt{m\_pd.h}.
\end{itemize}
\item Existe um arquivo de cabeçalho com constantes, tipos, estruturas de
dados e assinaturas de funções: \texttt{m\_pd.h}.
\end{itemize}
\end{frame}


\begin{frame}[fragile]{Compilação}
\begin{lstlisting}
EXTNAME=meu_external
cc -DPD -fPIC -Wall -o ${EXTNAME}.o -c ${EXTNAME}.c
ld -shared -lc -lm -o ${EXTNAME}.pd_linux ${EXTNAME}.o
rm ${EXTNAME}.o
\end{lstlisting}
O Pure Data procura por objetos com nomes diferentes em cada sistema:
\begin{itemize}
  \item \texttt{meu\_external.\enfase{pd\_linux}} (GNU/Linux).
  \item \texttt{meu\_external.\enfase{pd\_irix5}} (Irix 5).
  \item \texttt{meu\_external.\enfase{pd\_darwin}} (Mac OS X).
  \item \texttt{meu\_external.\enfase{dll}} (MS Windows).
\end{itemize}
\end{frame}


\begin{frame}[fragile]{Arquivos de ajuda}
Um arquivo de ajuda é um patch do Pure Data com:
\begin{itemize}
\item Um nome informativo: \texttt{meu\_external\enfase{-help}.pd}.
\item Instruções de uso.
\item Exemplos de utilização.
\end{itemize}
\vspace{2em}
A seguinte função associa um arquivo de ajuda a uma classe de \external:
\begin{lstlisting}
class_sethelpsymbol(meu_external_class, gensym("meu_external_class-help"));
\end{lstlisting}
\end{frame}


\begin{frame}{Utilizando \externals}
Passos para utilizar um \external no Pure Data:
\begin{enumerate}
\item Escreva um arquivo \texttt{.c} com as funções, classes e métodos.
\item Compile o código-fonte para criar uma biblioteca compartilhada.
\item Informe ao Pure Data o caminho para o \external através de uma das
opções abaixo:
  \begin{itemize}
    \item Na criação do objeto, insira o caminho completo (relativo ou
    absoluto) para o objeto do \external.
    \item Na linha de comando, utilize a opção \enfasel{\texttt{-path
    <caminho>}}.
    \item Na interface gráfica, acesse a opção \enfasel{\texttt{File} $\rightarrow$
    \texttt{Path...}}.
  \end{itemize}
\item Crie um objeto na interface gráfica do Pd com o nome do arquivo do \external, sem a extensão.
\end{enumerate}
\end{frame}


\begin{frame}{Utilizando \externals}
\begin{figure}[h!]
  \centering
  \includegraphics[width=0.7\textwidth]{../images/path}
  \caption{Adicionando o diretório de um \external ao caminho de busca do Pure Data.}
  \label{fig:search-path}
\end{figure}
\begin{itemize}
\item
\end{itemize}
\end{frame}

% ----------------------------------------------------------------------------
% O básico de um \external
% ----------------------------------------------------------------------------

\chapter{O básico de um \external}

Escrever um \external significa seguir as recomendações da API. Peço ao leitor
bastante paciência pois este tutorial pretende andar um pouco devagar para
mostrar os passos da escrita de um \external.

\section{Um \external simples}

Como dissemos anteriormente, a arquitetura do Pure Data é organizada de acordo
com o paradigma de orientação a objetos: cada objeto gráfico do Pure Data
corresponde a uma instância de uma classe. Neste sentido, um \external está
associado a um conjunto de estruturas de dados que representam classes em C.
Para cada classe é necessário haver métodos de instanciação, destruição,
processamento de sinais, tratamento de mensagens, etc.

A infraestrutura mínima para o funcionamento de um external (de nome, digamos,
\texttt{<external>}) consiste em uma estrutura de dados para a representação
de uma classe, que deve ter nome \texttt{t\_<external>}, e dois métodos
obrigatórios, chamados \texttt{<external>\_setup()} e
\texttt{<external>\_new()}. Note que a convenção de nomes utilizada no Pure
Data é de que toda função deve ser nomeada da forma
\texttt{<contexto>\_<funcao>()}.

A estutura de dados que representa uma classe do Pure Data deve
obrigatoriamente possuir o primeiro atributo do tipo \texttt{t\_object}, no
qual é armazenado o objeto criado no momento da instanciação.  Outros
atributos podem ser adicionados a esta estrutura de maneira que cada instância
da mesma classe possua os atributos necessários para seu funcionamento. Uma
classe que acessa um arquivo, por exemplo, pode possuir como atributos uma
string para guardar o caminho e um inteiro para guardar o descritor do
arquivo.

Um exemplo de estrutura de dados para representação de uma classe chamada
\texttt{example1} consiste no seguinte:

\vspace{1em}
\begin{lstlisting}
static t_class *example1_class;

typedef struct _example1 {
  t_object x_obj;
} t_example1;
\end{lstlisting}

Sempre que um \external é carregado pelo Pure Data, o método de nome
\texttt{<external>\_setup()} é executado. No exemplo dado acima, o Pure
Data irá procurar, no arquivo binário \texttt{example1.pd\_linux} que contém
a biblioteca compartilhada, o método de nome \texttt{example1\_setup(void)}.
Este método é utilizado para realizar a inicialização da classe, informando ao
Pure Data da existência de uma nova classe no sistema e associando a ela os
métodos de instanciação e destruição, além de outras informações:

\vspace{1em}
\begin{lstlisting}
void example1_setup(void) {
  example1_class = class_new(
    gensym("example1"),         // Nome simbolico
    (t_newmethod) example1_new, // Construtor
    0,                          // Destrutor
    sizeof (t_example1),        // Tamanho dos atributos
    CLASS_NOINLET,              // Flags da classe
    0                           // Tipos dos argumentos
  );
}
\end{lstlisting}

Dentro do método \texttt{<external>\_setup()} não há limite para o número de
classes a definir, de forma que é possível definir apenas uma classe (como no
exemplo 1) ou uma biblioteca inteira com várias classes (como no exemplo 3).
A introdução de uma nova classe no sistema é realizada pela função
\texttt{class\_new()}. São parâmetros da função \texttt{class\_new()}:

\begin{itemize}
\item Nome simbólico da classe.
\item Método construtor de um objeto.
\item Método destrutor de um objeto.
\item Tamanho do espaço de dados dos atributos de um objeto.
\item Flags que definem a representação gráfica do objeto.
\item Tipos dos parâmetros a serem passados para o construtor quando da
      instanciação de um objeto (veja o próximo capítulo).
\end{itemize}

É necessário terminar a lista de tipos de parâmetros com um número inteiro 0,
para indicar ao Pure Data que a lista de tipos terminou. Consulte a
documentação da função \texttt{class\_new()} para mais
detalhes\footnote{http://pdstatic.iem.at/externals-HOWTO/node9.html\#SECTION00092100000000000000}.

O método \texttt{<external>\_new()}, que foi associado como método de
instanciação de objetos na chamada de \texttt{class\_new()}, realiza a
instanciação de objetos propriamente dita. Neste método, além da instanciação
de um novo objeto através da função \texttt{pd\_new()}, é possível definir os
valores dos atributos da estrutura de dados da classe e também inicializar
quaisquer outros contextos que sejam necessários, como por exemplo abrir
arquivos, preencher vetores, alocar memória, etc.

\vspace{1em}
\begin{lstlisting}
// Construtor da classe
void * example1_new(void) {
    t_example1 *x = (t_example1 *) pd_new(example1_class);
    return (void *) x;
}
\end{lstlisting}

Após a criação da estrutura de dados dos métodos da forma mencionada acima, a
compilação realizada da forma descrita na seção \ref{sec:compiling}, e a
criação do objeto no Pure Data como descrito na seção \ref{sec:using}, o
resultado pode ser visto na figura \ref{fig:example01working}.

\begin{figure}[h!]
  \centering
  \includegraphics[width=0.7\textwidth]{example1}
  \caption{Nosso primeiro \external do PD. Ainda inútil. :-$\left.\right)$}
  \label{fig:example01working}
\end{figure}

\section{Uma biblioteca simples}

Um mesmo método \texttt{<external>\_setup()} pode definir várias classes
diferentes. A isto damos o nome de biblioteca. Neste cenário, o método
\texttt{<external>\_setup()} possui o mesmo nome do arquivo com a biblioteca,
mas cada classe podem ter um nome diferente (veja o exemplo 3).

\vspace{1em}
\begin{lstlisting}
void example3_setup(void) {
  post("Initializing my library");

  myobj1_class = class_new(
    gensym("myobj1"),
    (t_newmethod) myobj1_new, // Constructor
    0,
    sizeof (t_myobj1),
    CLASS_NOINLET,
    0);
  class_sethelpsymbol(myobj1_class, gensym("myobj1-help"));

  myobj2_class = class_new(
    gensym("myobj2"),
    (t_newmethod) myobj2_new, // Constructor
    0,
    sizeof (t_myobj2),
    CLASS_NOINLET,
    0);
  class_sethelpsymbol(myobj2_class, gensym("myobj2-help"));
}
\end{lstlisting}

Se o arquivo foi preenchido corretamente, compilado corretamente e adicionado
ao caminho do PureData, teremos o resultado visto na figura \ref{fig:exemplo3}.

\begin{figure}[h!]
	\centering
	\includegraphics[width=0.7\textwidth]{example3}
	\caption{Nosso segundo \external do PD. Ainda inútil. :-$\left.\right)$}
        \label{fig:exemplo3}
\end{figure}


Dentro do Pure Data, um clique com o botão direito em um objeto abre um menu
no qual uma das opções é \texttt{Ajuda}. Quando esta opção é selecionada, o
Pure Data abre um patch associado ao objeto, que deve conter instruções e
exemplos de uso. Por padrão, o Pure Data procura um arquivo com o mesmo nome
que o external (acrescido da extensão \texttt{-help.pd}) no diretório padrão de
documentação (\texttt{doc/5.reference}). Para associar um arquivo diferente do
padrão, basta utilizar a função \texttt{class\_sethelpsymbol}:

\vspace{1em}
\begin{lstlisting}
class_sethelpsymbol(myclass_class, gensym("my_class-help"));
\end{lstlisting}

Um objeto pode ainda ter outros nomes (\emph{aliases}). Para definir isto
podemos utilizar a função \texttt{class\_addcreator()}. Veja o exemplo:

\vspace{1em}
\begin{lstlisting}
class_addcreator((t_newmethod)medusa_new, gensym("med"), 0);
\end{lstlisting}

\section{Variáveis globais}

É possível utilizar variáveis globais para armazenar dados de um \external.
Estas variáveis são visíveis para todas as intâncias de objetos do \external e
todas podem alterar seus valores. Isto pode ser útil ou um desastre (veja o
exemplo16). Por exemplo, cada instância do \external \texttt{example16}
definido a partir do código a seguir incrementa em uma unidade o valor do
contador, como pode ser visto na figura \ref{fig:counter}:

\vspace{1em}
\begin{lstlisting}
int count = 0;

void * example16_new(void) {
    t_example16 *x = (t_example16 *) pd_new(example16_class);
    post("Counter value: %d",count);
    count++;
    return (void *) x;
}
\end{lstlisting}

\begin{figure}[h!]
  \centering
  \includegraphics[width=0.7\textwidth]{example16}
  \caption{Repare na saída da janela principal.}
  \label{fig:counter}
\end{figure}

Caso isto não seja desejável, o ideal é incluir as variáveis dentro da
estrutura do objeto. Assim, neste exemplo cada instância terá seu próprio
contador:

\vspace{1em}
\nopagebreak{
\begin{lstlisting}
static t_class *example_class;

typedef struct _example {
    t_object x_obj;
    t_int counter;
} t_example;

void * example_new(void) {
    t_example *x = (t_example *) pd_new(example_class);
    post("Counter value: %d",x->counter);
    x->counter++;
    return (void *) x;
}
\end{lstlisting}
}

% ----------------------------------------------------------------------------
% OS TIPOS DE DADOS DO PD
% ----------------------------------------------------------------------------

\chapter{Os tipos de dados do PD}

Uma vez que o Pure Data é utilizado em diversas plataformas, muitos tipos
comuns de variáveis, como \texttt{int}, são redefinidos. Para escrever um
\external que seja portável para qualquer plataforma, é razoável que você
utilize os tipos providos pelo Pure Data. Como dissemos na seção
\ref{sec:organizacao}, para escrever um \external, é necessário incluir o
arquivo \texttt{m\_pd.h} que possui definições de constantes (versão do Pure
Data, sistema operacional, compilador, etc), estruturas, assinaturas de
funções e tipos de dados.

Existem muitos tipos predefinidos que devem fazer a vida do programador mais
simples. Em geral, os tipos do pd têm nome iniciado por \texttt{t\_}.

\begin{center}
\begin{tabular}{|l|l|}
\hline
pd-type & description \\
\hline
\texttt{t\_atom} & \texttt{atom} \\
\texttt{t\_float} & \texttt{floatingpoint value} \\
\texttt{t\_symbol} & \texttt{symbol} \\
\texttt{t\_gpointer} & \texttt{pointer (to graphical objects)} \\
\texttt{t\_int} & \texttt{integer value} \\
\texttt{t\_signal} & \texttt{structure of a signal} \\
\texttt{t\_sample} & \texttt{audiosignal-value (floatingpoint)} \\
\texttt{t\_outlet} & \texttt{outletof an object} \\
\texttt{t\_inlet} & \texttt{inlet of an object} \\
\texttt{t\_object} & \texttt{object-interna} \\
\texttt{t\_class} & \texttt{a pd-class} \\
\texttt{t\_method} & \texttt{class-method} \\
\texttt{t\_newmethod} & \texttt{pointer to a constructor (new-routine)} \\
\hline
\end{tabular}
\end{center}

\section{Símbolos}

Um símbolo corresponde a um valor constante de uma \emph{string}, ou seja, a
uma sequência de letras que formam uma palavra.

Cada símbolo é armazenado em uma tabela de busca por razões de performance. A
função \texttt{gensym(char *)} procura por uma string em uma tabela de busca e
retorna o endereço daquele símbolo. Se a string não foi encontrada na tabela,
um novo símbolo é adicionado.


\texttt{t\_symbol} e \texttt{*gensym}

\section{Mensagens}

Dados que não correspondem a áudio são distribuídos via um sistema de
mensagens. Cada mensagem consiste em um ``seletor" e uma lista de átomos.

\subsection{Átomos}

Existem três tipos de átomos:

\begin{itemize}
\item \texttt{A\_FLOAT}: um valor numérico (de ponto flutuante).
\item \texttt{A\_SYMBOL}: um valor simbólico (string).
\item \texttt{A\_POINTER}: um ponteiro.
\end{itemize}

Valores numéricos são sempre considerados valores de ponto flutuante
(\texttt{t\_float}), mesmo que possam ser exibidos como valores inteiros.

Átomos do tipo \texttt{A\_POINTER} não são muito importantes (para \externals
simples).
Since pd is used on several plattforms, many ordinary types of variables, like int, are re-defined. To write portable code, it is reasonable to use types provided by pd.

O tipo de um átomo \texttt{a} é armazenado no elemento da estrutura
\texttt{a.a\_type}.

\subsection{Seletores}

Um seletor é um símbolo que define o tipo de uma mensagem. Existe cinco
seletores pré-definidos:

\begin{itemize}
  \item \texttt{bang}: rotula um gatilho de evento. Uma mensagem de
    \texttt{bang} consiste somente do seletor e não contém uma lista de átomos.
  \item \texttt{float} rotula um valor numérico. A lista de uma mensagem
    \texttt{float} contém um único átomo de tipo \texttt{A\_FLOAT}.
  \item \texttt{symbol} rotula um valor simbólico. A lista de uma mensagem
    \texttt{symbol} consiste em um único átimo do tipo \texttt{A\_SYMBOL}.
  \item \texttt{pointer} rotula um valor de ponteiro. A lista de uma mensagem do
    tipo \texttt{pointer} contém um único átimo do tipo \texttt{A\_POINTER}.
  \item \texttt{list} rotula uma lista de um ou mais átomos de tipos arbitrários.
\end{itemize}

Uma vez que os símbolos para estes seletores são utilizados com frequência,
seu endereço na tabela de símbolos pode ser utilizado diretamente, sem a
necessidade da utilização de \texttt{gensym}:

\begin{center}
\begin{tabular}{|l|l|l|}
\hline
  selector & lookup-routine & lookup-address \\
\hline
  \texttt{bang} & \texttt{gensym("bang")} & \texttt{\&s\_bang} \\
  \texttt{float} & \texttt{gensym("float")} & \texttt{\&s\_float} \\
  \texttt{symbol} & \texttt{gensym("symbol")} & \texttt{\&s\_symbol} \\
  \texttt{pointer} & \texttt{gensym("pointer")} & \texttt{\&s\_pointer} \\
  \texttt{list} & \texttt{gensym("list")} & \texttt{\&s\_list} \\
  \texttt{--}  (signal) & \texttt{gensym("signal")} & \texttt{\&s\_symbol} \\
\hline
\end{tabular}
\end{center}

Outros seletores também podem ser utilizados. A classe receptora tem que
prover um médodo para um seletor específico ou para \texttt{anything}, que
corresponde a qualquer seletor arbitrário.

Mensagens que não possuem seletor explícito e começam com um valor numérico
são reconhecidas automaticamente como mensagens \texttt{float} (se consistirem
de apenas um átomo) ou como mensagens \texttt{list} (se forem compostas de
diversos átomos).

Por exemplo, as mensagens \texttt{12.429} e \texttt{float 12.429} são
idênticas. Da mesma forma, as mensagens \texttt{list 1 para voce} é idêntica a
\texttt{1 para voce}.


\section{Conclusão}

\begin{frame}{Referências}
\begin{itemize}
\item Repositório oficial de externals do Pure Data\footnote{http://pure-data.svn.sourceforge.net/viewvc/pure-data/trunk/externals}.
\item Tutorial do IOHannes\footnote{http://iem.at/pd/externals-HOWTO/pd-externals-HOWTO.pdf}.
\item Código fonte do Pure Data\footnote{http://pure-data.git.sourceforge.net/}.
\end{itemize}
\end{frame}


\begin{frame}{Dúvidas?}
{Obrigado!}
\texttt{http://www.ime.usp.br/$\sim$fls}

\texttt{https://github.com/flschiavoni/pd-external-tutorial}

\texttt{http://compmus.ime.usp.br/}
\end{frame}



\end{document}
